\documentclass{beamer}

%%%%%%%%%%%%%%%%%%%%%%%%%%%%%%%%%%%%

%reference

\usepackage[english]{babel}
\usepackage[backend=biber,style=alphabetic-verb,sorting=none]{biblatex}

\addbibresource{reference.bib} %Import the bibliography file

%%%%%%%%%%%%%%%%%%%%%%%%%%%%%%%%%%%% 

\usetheme{default}

%%%%%%%%%%%%%%%%%%%%%%%%%%%%%%%%%%%% 

%commands

\def\T{\mathbb{T}}
\def\Z{\mathbb{Z}}
\def\R{\mathbb{R}}
\def\C{\mathbb{C}}

\renewcommand{\hat}\widehat
%%%%%%%%%%%%%%%%%%%%%%%%%%%%%%%%%%%%

% \title{Bourgain's transference theorem: HT implies UMD}

\title{Introduction to Abstract Harmonic Analysis}

\author{Report the name of the dish}
% \author{Hao Fan}

\date{\today}
%%%%%%%%%%%%%%%%%%%%%%%%%%%%%%%%%%%%


\begin{document}

\begin{frame}
  \titlepage
\end{frame}

\begin{frame}
\frametitle{Notation}

\begin{itemize}
  \item The \emph{Direct Product of Abelian Groups} of $\{G_i\}_{i\in I}$ is defined as the set Cartesian product  $\prod G_i$ with componment-wise operation.
  \item The \emph{Direct Sum of Abelian Groups} of $\{G_i\}_{i\in I}$ is defined as the following subgroup of $\prod G_i$:
        \[\bigoplus G_i:=\Bigl\{x\in\prod G_i \left|\right. \sharp\{i\in I: x_i\ne0\}<\infty\Bigr\}.\]
\end{itemize} 

\begin{example}
    The torus/unit sphere $\mathbb{T}$ is a compact topological group and we consider $G=\prod \mathbb{T}$, also a compact topological group.
\end{example}

\begin{example}
    The integer $\mathbb{Z}$ is a LCH group and we consider $\Gamma=\bigoplus\mathbb{Z}$, also a LCH group.
\end{example}
\end{frame}

\begin{frame}%Haar measure
\frametitle{Haar Measure}

\begin{definition}
A \emph{Haar measure} on a LCH topological group $G$ is a non-zero Radon measure $\mu$ on $G$ that satisfies the following properties:
\begin{itemize}
  \item $\mu(A) > 0$ for every non-empty open set $A$,
  \item $\mu(K) < \infty$ for every compact set $K$,
  \item $\mu(gE) = \mu(E)$ for all Borel sets $E$ and $g \in G$.
\end{itemize}
\end{definition}

\begin{Theorem}[Haar, 1933]
Let $G$ be a LCH group. Then there exists a Haar measure on $G$, which is unique up to a constant.
\cite*[Chapter~2]{folland2016course}
\end{Theorem}

\begin{example}
  We check $\R$, $\Z$, and $\T$.
\end{example}

\end{frame}

% \begin{frame}
% \frametitle{Transference Trick}
% \begin{theorem}
%   Suppose there are Banach spaces and linear maps satisfying the following commutative diagrams. Then isomophisms 
% \end{theorem}
% \end{frame}

\begin{frame}[allowframebreaks]
  \frametitle{Fourier Analysis On LCH Group}
We start from the simple case, on the group $\Z$:
\begin{definition}[Fourier Transform]
  Let $f\in\ell_1(\Z)$. The \emph{classical Fourier fransform} of this function is the function on $\T$ defined by the formula
    \[\hat{f}(e^{i\theta}):=\int_{\Z}f(n)e^{in\theta}d\sharp(n)=\sum_{n\in\Z}f(n)e^{in\theta}.\]
\end{definition}

\begin{theorem}
  View $\ell_1=\ell_1(\mathbb Z)$, equipped with involution $a^\ast:=\overline{a}$ and convolution, as a $\ast$-Banach algebra.
  Then $\sigma(\ell_1):=\operatorname{Hom}_{\mathsf{Alg}}(\ell_1,\C)$ can be identified with $\T$ in such a way that the Gelfand transform
  on $\ell_1$ becomes
    \[\hat{f}(e^{i\theta})=\sum_{n\in\Z} f_n e^{in\theta}.\]
\end{theorem}

In general, $L_1(G)$ with convolution is a Banach $\ast$-algebra, and hence one can define its spectrum $\sigma (L_1(G))$.\cite*[Chapter~4.1]{folland2016course}
  \begin{definition}[Dual Group]
    For a LCH group $G$, the dual group is
      \[\hat{G}:=\operatorname{Hom}_{\mathsf{TopGrp}}(G,\T)=\sigma(L_1(G)).\]
  \end{definition}
    
    \begin{example}
      For usual groups: $\hat{\R}=\R$, $\hat{\Z}=\T$ and $\hat{\T}=\Z$. As you should verify by $\operatorname{Hom}_{\mathsf{TopGrp}}(G,\C)$.
    \end{example}

    \begin{block}{Exercise}
      What is the dual group of $G:=\T^{\Z_{\geq0}}$?
    \end{block}

    \newpage

    Now we can define Fourier transform.

  \begin{definition}
    Associate to $\xi\in\hat{G}$ the functional  
    \[f\mapsto \bar{\xi}(f)=\xi^{-1}(f)=\int_G \overline{\langle x,\xi\rangle}f(x) dx.\]
      The Fourier transform on G is defined as the Gelfand trnsform $L_1(G)\to C(\hat{G})$,
      \[\mathcal{F}(f)(\xi)=\hat{f}(\xi):=\xi^{-1}(f)=\int_G \overline{\langle x,\xi\rangle}f(x)dx.\]
    \end{definition}
    
    \begin{example}
      For usual groups: $\hat{\R}=\R$, $\hat{\Z}=\T$ and $\hat{\T}=\Z$. Convince yourself again since you should have studied Fourier transform on them.
    \end{example}
    

  \begin{theorem}[Pontrjagin Duality]
    The map $\Phi\colon G\to \hat{\hat{G}},x\mapsto \Phi(x)$ where $\langle\xi,\Phi(x)\rangle=\langle x,\xi\rangle\forall \xi\in\hat{G}$, is an isomophism of topological groups.
  \end{theorem}

  \begin{theorem}
    If $G$ is compact and $|G|=1$, $\hat{G}$ is an orthonormal basis for $L_2(G)$.\cite*[Corollary~4.27]{folland2016course}
  \end{theorem}
This theorem ensures define multipliers from $\ell_\infty(\hat{G})$.
  \begin{example}
    For the compact group $\T$, $\hat{\T}=\Z=\{e_k\}_{k\in\Z}$, the Fourier basis is surely an orthonormal basis for $L_2(\T)$.
  \end{example}
\end{frame}

\begin{frame}
If you have an algebra, you should care its ideals.
  \begin{definition}
    Let $A$ be a commutative Banach algebra with spectrum $\sigma(A)$.
% \[\begin{tikzcd}
%   \{closed ideals in A\} \arrow[rr, "\nu", shift left=2] &  & closed subsets of \sigma(A) \arrow[ll, "\iota", shift left=2] \\
%   J \arrow[rr, maps to]                                  &  & {}                                                            \\
%   {}                                                     &  & N \arrow[ll, maps to]                                        
%   \end{tikzcd}\]
  \end{definition}
  
  \begin{theorem}
    The maximal ideals in the ring $C[0,1]$ are exhausted by $\ker\hat{p}, p\in [0,1]$.
  \end{theorem}

  \begin{theorem}[Wiener's Theorem]
    If ${J}$ is a closed ideal in $L_1(G)$ and $\nu ({J})=\varnothing$, then ${J}=L_1(G)$.
  \end{theorem}
This is analogue of Hilbert's weak Nullstellenstaz.
  \begin{theorem}[Hilbert's weak Nullstellenstaz]
    Let $k$ be an algebraically closed field, $J\subseteq k[x_1,x_2,\ldots,x_n]$ an ideal. Then $\nu(J)=\varnothing$ iff $J=k[x_1,x_2,\ldots,x_n]$, i.e. $1\in J$.
  \end{theorem}
\end{frame}

\begin{frame}%[allowframebreaks]
        \frametitle{References}
        \printbibliography
\end{frame}
\end{document}