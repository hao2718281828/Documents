\section{HT implies UMD }\label{sec-HT to UMD}

\subsection{Notation}
\begin{itemize}
    \item $\Gamma:=\bigoplus_{k\geq0}\mathbb{Z}$, $G:=\prod_{k\geq0}\mathbb{T}$, if there is no specific clarification;
    \item $\Gamma$ is equiped with the anti-lexicographical order, that is, $\forall n,m\in\Gamma$, we define $n\geq m$ if $\exists N\in \mathbb{N}$ such that $n_j\geq m_j$ holds for all $j\geq N$. We define $n>m$ if and only if $n\geq m$ and $n\ne m$.
    \item $\operatorname{sign}\colon\Gamma\to\{-1,0,1\},n\mapsto\operatorname{sign}$ is defined as the rule on $\mathbb{Z}$.
    \item  $supp(n):=\{k\geq0| n_k\ne0\}$,
    \item $\gamma_n\colon G\to \mathbb{T}, z\mapsto \prod z_k^{n_k}$,
    \item Product of two sequences/functions is defined in the sense of pointwise product. 
\end{itemize}


\subsection{Main part}
\begin{definition}[Haar measure, Dual group]
A
\end{definition}
Here is a technical lemma will be used latter.
\begin{lemma}
    For any finite subset $A\subseteq \Gamma$, there is an $m\in\Gamma$ such that
        \[\forall n \in A\setminus\{0\}\quad \operatorname{sign}(\langle n,m\rangle)=\operatorname{sign}(n).\]
    More generally, for any sequence of signs $(\varepsilon_k)_{k\geq0}$ there is $m\in\Gamma$ such that
        \[\forall n \in A\setminus\{0\}\quad \operatorname{sign}(\langle \varepsilon n,m\rangle)=\operatorname{sign}(\varepsilon n).\]
\end{lemma}
\begin{proof}
    The second assertion follows from the first one applied to $A^{\prime}:=\{(\varepsilon n)| n\in A\}$, which is finite since for each $n\in A$, ${\varepsilon n| \varepsilon\text{ is an arbitrary sign}}$ contains 
    at most $2^{\sharp supp(n)}$ distinct sequences, which ensures that $\sharp A^{\prime}\leq 2^{\max_{n\in A}\sharp supp(n)}\sharp A$ is finite. Now it suffices to prove the first one.
    \par Let $W_K:=\{n\in\Gamma|n_K\ne0,n_j=0\forall j>K\}$ be the set of ‘words’ of length exactly $K + 1$. The idea is simply that for any $n \in W_K$ we have
        \[  \lim_{m_K\to\infty} \operatorname{sign}(\langle n,m\rangle)=\operatorname{sign}(n_K)\cdot\infty=\operatorname{sign}(n)\cdot\infty.\]
    By induction on K we will prove that for any $A\subseteq \cup_{j=0}^{K}W_j$ there is $m\in \cup_{j=0}^{K}W_j$ such that
    \begin{equation}\label{eqn:sign property}
        \forall n\in A\quad \operatorname{sign}(\langle n,m\rangle)=\operatorname{sign}(n).
    \end{equation}
    The case $K = 0$ is trivial: take $m=(1,0,0,\ldots)\in W_0$. Assume that we have proved this for any $A\subseteq \cup_{j=0}^{K-1}W_j$ and let 
    us prove it for $A\subseteq \cup_{j=0}^{K}W_j$. By the induction hypothesis there is $(m_0,\ldots,m_{K-1},0,\ldots)$ such that $\operatorname{sign}(\langle n,m\rangle)=\operatorname{sign}(n)$ 
    for any $n\in A\cap( \cup_{j=0}^{K-1}W_j )$. Consider now any $n \in A \cap W_K$ so that $n_K\ne0$. Then choosing $m_K>0$ large enough ensures that $\operatorname{sign}(\langle n,m\rangle)=\operatorname{sign}(n)$, 
    (which is possible as the limit mentioned above) and since $A$ is finite, we can achieve this for any $n\in A\cap W_K$. Then $m=(m_0,\ldots,m_{K-1},m_{K},0,\ldots)$ satisfies~\eqref{eqn:sign property}, completing the induction argument.
\end{proof}

\begin{remark}
    Filter limit?
\end{remark}

\begin{definition}[Multiplier]
    Given a bounded function $\varphi\colon \Gamma\to\mathbb{C}$ on a discrete group $\Gamma$(we refer to $\varphi$ as
    a ‘multiplier’), we will always denote by $M_\varphi\colon L_2(\widehat{\Gamma})\to L_2(\widehat{\Gamma})$ the corresponding 
    multiplier operator on $L_2(\widehat{\Gamma})$ defined by $M_\varphi(\gamma_n)=\varphi(n)\gamma_n$ for any $\gamma_n\in\Gamma$. 
\end{definition}

\begin{lemma}[Transference argument]\label{lem: trans}
    Let $B$ be a Banach space. Let $1\leq p<\infty$. Let $\varphi\colon \mathbb{Z}\to\mathbb{C}$ be a multiplier. For
    any $m\in \Gamma$ we define the multiplier $\varphi_m\colon\Gamma\to\mathbb{C}$ by
        \[\forall n\in\Gamma\quad \varphi_m(n)=\varphi(\langle n,m\rangle). \]
    We have then:
        \[  \|M_\varphi\|_{\mathcal{B}(L_p(G;B))}\leq\|M_\varphi\|_{\mathcal{B}(L_p(\mathbb{T};B))}.\]
\end{lemma}
\begin{proof}
By homogeneity we may assume $\|M_\varphi\|_{\mathcal{B}(L_p(\mathbb{T};B))}\leq1$. By Fubini we also have $\|M_\varphi\|_{\mathcal{B}(L_p(\mathbb{T});L_p(G;B))}\leq1$:

Fix $m \in \Gamma$. For any $z = (z_k) \in \Gamma$ and any $w \in \mathbb{T}$ we denote
    \[
        w_{\bullet}z = (w^{m_k} z_k)_{k \geq 0}.
    \]
To any $f \in L_p(G; B)$ we associate $\widetilde{f} \in L_p(\mathbb{T}, L_p(G; B))$ defined by
    \[
        \widetilde{f}(w)(z) := f(w_{\bullet}z).
    \]
This is an isometric embedding from $L_p(G;B)$ to $L_p(\mathbb{T};L_p(G;B))$:
    \begin{align*}
        \int_{\mathbb{T}}\|\widetilde{f}(w,\ )\|_{L_p(G;B)}^{p} dw
        &=\int_{\mathbb{T}}\int_{G} \|f(w_{\bullet}z)\|_B^p dz dw\\
        &=\int_{\mathbb{T}}\int_{G} \|f(w_{\bullet}z)\|_B^p dz dw\\
        &=\int_{\mathbb{T}}\int_{G} \|f(z)\|_B^p dz dw\\
        &=\int_{G}\int_{\mathbb{T}} \|f(z)\|_B^p dz dw\\
        &=\int_{G} \|f\|_B^p d\mu_G,
    \end{align*}
i.e. $\|\widetilde{f}\|_{L_p(\mathbb{T};L_p(G;B))}=\|f\|_{L_p(G;B)}$. 

Above all: the following diagram commutes and two arrows $\hookrightarrow$ are isometric embeddings $f\mapsto\widetilde{f}$.
    \[\begin{tikzcd}
        L_p(G;B) \arrow[d, "\varphi_m"'] \arrow[r, hook] & L_p(\mathbb{T};L_p(G;B)) \arrow[d] \\
        L_p(G;B) \arrow[r, hook]                         & L_p(\mathbb{T};L_p(G;B))          
        \end{tikzcd}\]
    \end{proof}
By Fubini, we have $L_p(G\times\mathbb{T};B)=L_p(\mathbb{T};L_p(G;B))$. The former space works better with weak $L_1$ norm than the latter space.
    \begin{remark}
        An analogous result holds for multipliers of weak-type $(1,1)$. We have:
        \[  \|M_\varphi\|_{\mathcal{B}(L_1(G;B),L_{1,\infty}(G;B))}\leq\|M_\varphi\|_{\mathcal{B}(L_1(\mathbb{T};B),L_{1,\infty}(\mathbb{T};B))}.\]
    \end{remark}
    Indeed, we can use the elementary fact.
    \begin{lemma}
        For any function $f$ on a product of measure spaces such as $G\times\mathbb{T}$ we have
        \[  \|f\|_{L_{1,\infty}(G\times\mathbb{T})}\leq\int_{G}\|f(z,\cdot)\|_{L_{1,\infty}(\mathbb{T})}d\mu(z).     \]
    \end{lemma}
    \begin{proof}
        For the product measure and any $\lambda>0$, we have
        \begin{align*}
            \lambda(\mu_{G}\times\mu_{\mathbb{T}})(\{|f|>\lambda\})
              &=\int_{G\times\mathbb{T}}\lambda\mathbf{1}_{\{|f|>\lambda\}}d(\mu_G\times\mu_{\mathbb{T}})
            \\&=\int_{G}d\mu_G(z)\int_{\mathbb{T}}\lambda\mathbf{1}_{\{|f(z,-)|>\lambda\}}(w)d\mu_{\mathbb{T}}(w)
            \\&= \int_{G}\lambda\mu_{\mathbb{T}}({\{|f(z,-)|>\lambda\}})d\mu_G(z)
            \\&\leq \int_{G}\|f(z,\cdot)\|_{L_{1,\infty}(\mathbb{T})}d\mu(z).\qedhere
        \end{align*}
    \end{proof}
    \begin{lemma}
        We have an isometric embedding $L_{1,\infty}(G;B)\hookrightarrow L_{1,\infty}(G\times\mathbb{T};B)$
    \end{lemma}
    \begin{proof}
        For any $\lambda>0$, we have:
            \begin{align*}
                \lambda(\mu_{G}\times\mu_{\mathbb{T}})(\{\|\widetilde{f}\|_B>\lambda\})
                  &=\int_{G\times\mathbb{T}}\lambda\mathbf{1}_{\{\|\widetilde{f}\|_B>\lambda\}}d(\mu_G\times\mu_{\mathbb{T}})
                \\&=\int_{\mathbb{T}}\int_{G}\lambda\mathbf{1}_{\{z\in G\colon\|f(w_{\bullet}z)\|_B>\lambda\}}(z)d\mu_G(z)d\mu_{\mathbb{T}}(w)
                \\&=\int_{\mathbb{T}}\int_{G}\lambda\mathbf{1}_{\{z\in G\colon \|f(z)\|_B>\lambda\}}(\overline{w}_{\bullet}z)d\mu_{G}(z)d\mu_{\mathbb{T}}(w)
                \\&=\int_{\mathbb{T}}\int_{G}\lambda\mathbf{1}_{\{z\in G\colon \|f(z)\|_B>\lambda\}}(z)d\mu_{G}(z)d\mu_{\mathbb{T}}(w)
                \\&=\int_{G}\int_{\mathbb{T}}\lambda\mathbf{1}_{\{z\in G\colon \|f(z)\|_B>\lambda\}}(z)d\mu_{\mathbb{T}}(w)d\mu_G(z)
                \\&=\int_{G}\lambda\mathbf{1}_{\{z\in G\colon \|f(z)\|_B>\lambda\}}(z)d\mu_G(z)
                \\&=\lambda\mu_{G}(\{z\in G\colon \|f(z)\|_B>\lambda\}),
            \end{align*}
        which ensures what we want.
    \end{proof}
    The following diagram commutes. 
    \[\begin{tikzcd}
        L_1(G;B) \arrow[d, "\varphi_m"'] \arrow[r, hook] & L_1(G\times\mathbb{T};B) \arrow[d]  \\
        {L_{1,\infty}(G;B)} \arrow[r, hook]              & {L_{1,\infty}(G\times\mathbb{T};B)}
    \end{tikzcd}\]

    Two arrows $\hookrightarrow$ are isometric embeddings. Therefore, we know $\|\varphi_m\| \leq \|\varphi \| $.
%%%%%%%%%%%%%%%%%%%%%%%%%%%%%%%

We can now state Bourgain’s transference theorem:
\begin{theorem}\label{thm: Bourgain trans}
    Let $1 < p < \infty$. Let $B$ be any Banach space such that the Hilbert transform $H^{\mathbb{T}}$ is 
    bounded on $L_p(\mathbb{T};B)$. Recall $C_p^{HT}(B)=\|H^{\mathbb{T}}\|_{\mathcal{B}(L_p(G;B))}$. Then for any choice of signs 
    $\varepsilon_k=\pm1$ and for any finite martingale $f=\int fd\mu+\sum_{k\geq0}df_k$ in $L_p(G;B)$ we have
        \[  \|\sum_{k\geq0}\varepsilon_k H_k df_k\|_{L_p(G;B)}\leq C_p^{HT}(B)\|f\|_{L_p(G;B)}. \]
\end{theorem}

\begin{corollary}
    In the situation of Theorem~\ref{thm: Bourgain trans}, let $A\subseteq\mathbb{N}$ be a subset such
    that for each $n$ in $A$, the variable
\end{corollary}

\begin{corollary}
    For any Banach space $B$ and any $1 < p < \infty$ we have
        \[  C_p(B) \leq C_p^{HT}(B)^2.  \]
    In particular, HT implies UMD.
\end{corollary}

The following seems to be still open.
\begin{problem}
    Is there an absolute constant $K$ such that $C_p(B) \leq K C_p^{HT} (B)$ for
any $1 < p < \infty$ and any $B$?
\end{problem}

\appendix

Here we discuss about infinite tensor products of Hilbert spaces.

\section{For the infinite tensor product}
We start from tensor product of finite numbers of Hilbert spaces. In particular, we consider two Hilbert spaces.

\begin{definition}[Tensor product of two Hilbert spaces]
    
\end{definition}

\begin{definition}[Directed set, Direct system, Direct limit]
    Let $(I,\leq)$ be a partially ordered set. It is said to be a direct set if $\forall i,j\in I\exists k\in I$ such that $i\leq j, j\leq k$. A direct system in $\{A_i,(\phi_{i j})_{j\in I}\}_{i\in I}$ such that 
    \begin{itemize}
        \item for $i\in I$, $\phi_{ii}=1_{A_i}$;
        \item for $i\leq j\leq k$, $\phi_{j k}\circ\phi_{i j}=\phi_{i k}$.
    \end{itemize}
    A direct limit of a direct system is an object denoted by $\lim\limits_{\longrightarrow} A$ together with morphisms $\{\phi_i\to\lim\limits_{\longrightarrow} A\}_{i\in I}$, satisfying
    \begin{itemize}
        \item for all $\forall i\leq j\colon\phi_{j}\circ\phi_{i j}=\phi_i$;
        \item the universal property: for any object $C$ and any morphisms $\{\psi_i\colon A_i\to C\}_{i\in I}$ such that $\psi_j\circ\phi_{i j}=\psi_i$ for all $i\leq j$, there exists a unique morphism $\psi\colon \lim\limits_{\longrightarrow} A\to C$ such that $\psi_i=\psi\circ\phi_i$ for all $i\in I$.
    \end{itemize}
\end{definition}

\begin{remark}
    Every direct system of vector spaces and linear maps possesses a limit. 
    See \cite[\href{https://stacks.math.columbia.edu/tag/07N7}{Section 07N7}]{stacks-project}. The limit is unique up to a unique isometry that factors thorough $\{\phi_i\}_{i\in I}$.
\end{remark}

\begin{definition}[Infinite tensor product of pointed Hilbert spaces]
    For a family of "pointed" Hilbert spaces $\{(H_i,\xi_i)\}_{i\in I}$, in other words, $(H_i,\xi_i)$ is a pair of a Hilbert space $H_i$ together with a unit vector $\xi_i\in H_i$, the tensor product of $\{(H_i,\xi_i)\}_{i\in I}$ is defined by the direct limit of the directed system $\{H_F:=\otimes_{i\in F}H_i\}_{F\in\mathcal{F}(I)}$, where $\mathcal{F}:=\{F\in\mathcal{P}(I)|\sharp F<\infty\}$.
\end{definition}

Above all, we have:
    For $L_2(\mathbb{T})$, we have
        \[  L_2(G)=\bigoplus_{k\geq0}L_2(\mathbb{T}).   \]

To see this, take $I=\mathbb{N}$ and $H_i=L_2(\mathbb{T})$ for all $i\in\mathbb{N}$, $\xi_i=1\in H_i$. Then $L_2(G)$ satisfies the universal property (we will prove a simpler version of this: $L_2(\mathbb{T}^2)=L_2(\mathbb{T})\bigotimes L_2(\mathbb{T})$).

\begin{proof}
    Consider the fourier basis of $L_2(\mathbb{T})$, that is $\{e_k\}_{k\in\mathbb{Z}}$. 
    Now it follows from the fact that $\{e_k\otimes e_l\}_{k,l\in\mathbb{Z}}$ is an orthonormal family in $L_2(\mathbb{T}^2)$.
\end{proof}

