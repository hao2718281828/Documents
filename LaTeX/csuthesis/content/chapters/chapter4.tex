%!TEX root = ../../csuthesis_main.tex
\chapter{总结与展望}
\section{总结}

本文介绍了 Gundy 分解,包括经典的版本以及非交换的版本。具体地说:本文的第一章,简单地介绍了 Gundy 分解定理的相关历史。第二章中,我们给出了非交换版本的 Gundy 分解定理的内容及其证明:证明的关键是 Cuculescu 投影序列,这可以认为是停时的非交换类似物。第三章中,我们证明了经典版本的 Gundy 分解定理:证明的关键即停时的选取与应用。第四章则给出了 Gundy 分解定理的应用:包括经典的应用:鞅变换的有界性的等价刻画;也包括非交换的应用:证明非交换鞅变换的有界性,非交换均方函数的 Burkholder 不等式以及鞅差序列的几何性质。
\section{展望}
领域内有以下的问题尚待解决:
问题一:在非交换鞅变换的弱 $(1,1)$ 型不等式中,去掉关于子代数的交换性条件(在经典情形下交换性自动成立),也就是说要求 $ \xi_{k-1}\in \M_{k-1} $ 而不再要求 $ \xi_{k-1}\in \M_{k-1}\cap \M_{k}^\prime$ ,是否还可以证明非交换鞅变换的弱 $(1,1)$ 有界性?

问题二:关于小均方函数,现在已有的结果是之前陈述的 Narcisse 在 2007 年中的结果\cite{MR2319715-2007-Narcisse-Conditioned-square-functions}:存在三个适应序列 $y,z,w$ 使得 $dx_n=y+z+w$ 适合弱型估计
\[  \left|\!\left|\sum_{n=1}^N y_n\otimes e_{n,n}\right|\!\right|_{L^{1,\infty}(\M\overline{\otimes} \mathcal{B}(\ell^2_N))}+\left\| s_c(z) \right\|_{1,\infty}+\left\| s_r(w) \right\|_{1,\infty}\lesssim\|x\|_1. \]
鞅差序列自然地是适应序列,而适应序列不一定是鞅差序列;这里只做到了适应序列而非鞅差,所以自然的问题是:能不能做到 $a,b,c$ 均为鞅差的不等式刻画?这个问题甚至经典鞅论版本都仍然未解决。一种可能的解决方法是应用三角截断算子。

问题三:更一般的情形如何呢?这里的更一般指,将鞅变换推广为:考虑两个鞅 $x,y$,使得 $y$ 次微分从属于 $x$,我们希望刻画他们的均方函数的控制。这在经典情形下已经有 Burkholder 所做的结果:设 $y$ 是微分从属于 $x$ 的,即 $|dy_n|\leq|dx_n|$ 对所有 $n$ 成立,则有
\begin{align*}
    &\|y\|_{1,\infty}\leq \ 2\|x\|_1;\\
    &\|y\|_p\leq \ (p^\ast-1)\|x\|_p,\quad (1<p<\infty),
\end{align*}
这里 $p^\ast:=\max\{p,p^\prime\}$。这里的系数都已经是最佳的。而在非交换鞅论中,明显的复杂很多。已有的结果是焦勇,Osękowski,吴恋在文章\cite{JIAO2018216}中完成的如下估计:假设 $|dy_n|^2\leq |dx_n|^2$ 对所有 $n$ 成立,则
\[ \|y\|_p\leq c_p\|x\|_p. \]
对于 $1<p<2$ ,以及弱 $(1,1)$ 不等式,则需要新的定义:$y$ 微分从属于 $x$,如果对于任意 $n$ 和任意投影 $R\in\M_{n-1}$ 成立 $Rdy_nRdy_nR\leq Rdx_nRdx_nR$。假设 $y$ 微分从属于 $x$ ,则有
\begin{align*}
    &\|y\|_{1,\infty}\leq\ \|x\|_1;\\
    &\|y\|_{p}\leq\ c_p\|x\|_p,\quad 1<p<2.
\end{align*}
关于均方函数,自然的猜测是,假设 $y$ 弱微分从属于 $x$,则
\[ \|S_c(w)\|_{1,\infty}+\|S_r(z)\|_{1,\infty}\lesssim\|x\|_1, \]
其中 $w,z$ 为适应序列(甚至要求鞅差序列)使得 $y=w+z$。

%     问题四: 在经典鞅论中,我们有带权的鞅不等式,例如带权的 Doob 极大不等式:对于鞅 $f=(f_n)_{n\geq0}$ 以及权 $w$ 即非负可测函数 $w$ 为密度的测度,成立
%     \[\lambda w(f^\star_n>\lambda)\leq\int_{f^\star_n>\lambda}f_n w^\star {d}\mathbb{P},\forall \lambda>0\]
%     于是自然地会问,在非交换中,是否可以做出类似物?

% \chapter{公式与符号}

% 中南大学由原湖南医科大学、长沙铁道学院与中南工业大学于2000年4月合并组建而成。原中南工业大学的前身为创建于1952年的中南矿冶学院,原长沙铁道学院的前身为创建于1953年的中南土木建筑学院,两校的主体学科最早溯源于1903年创办的湖南高等实业学堂的矿科和路科。原湖南医科大学的前身为1914年创建的湘雅医学专门学校,是我国创办最早的西医高等学校之一。中南大学秉承百年办学积淀,顺应中国高等教育体制改革大势,弘扬以“知行合一、经世致用”为核心的大学精神,力行“向善、求真、唯美、有容”的校风,坚持自身办学特色,服务国家和社会重大需求,团结奋进,改革创新,追求卓越,综合实力和整体水平大幅提升。

% \LaTeX 的公式环境中符号样式符合 \TeX 默认的美国数学学会(AMS)的符号使用习惯,中文论文写作推荐遵循 GB/T 3102.11——1993《物理科学和技术中的数学符号》标准。这里我们给出一些 \LaTeX 中常用的符号表示。


% \section{\LaTeX 数学公式模式}

% \LaTeX 提供了两种数学公示的写作模式:内联模式和独显模式:

% \begin{itemize}
%     \item \textbf{内联模式}(inline mode),又称为行内模式,随文模式,将公式显示为段落的一部分。
%     \item \textbf{独显模式}(display mode),又称为行间模式,将公式用独立行展示出来,不再作为段落的一部分。
% \end{itemize}

% \subsection{内联模式}

% % TODO

% 键入如下定义符之一在段落中来使用内联模式书写数学公式符号:

% \begin{itemize}
%     \item \verb|\(...\)|
%     \item \verb|$...$|
%     \item \verb|\begin{math}...\end{math}|
% \end{itemize}

% \subsection{独显模式}

% 使用如下方式以独显模式表示数学公式:

% \begin{itemize}
%     \item \verb|\[...\]|
%     \item \verb|\begin{displaymath}...\end{displaymath}|
%     \item \verb|\begin{equation}...\end{equation}|
% \end{itemize}

% \textbf{公式插入示例如公式(\ref{E.example})所示。}

% \begin{equation}
% \gamma_{x}=
% \left\{
%   \begin{array}{lr}
%   0, & {\rm if}~~\;|x| \leq \delta \\
%   x, & {\rm otherwise}
%   \end{array}
% \right.
% \label{E.example}
% \end{equation}


% \newpage

