%!TEX root = ../../csuthesis_main.tex
\def\M{\mathcal{M}}
\def\T{\tau}
\def\E{\mathcal{E}}

\chapter{绪论}
\section{问题的历史背景}
经典的鞅论,出现在 18 世纪的法国,那时候鞅论是赌博策略的一部分。而最简单的赌博策略,来自如下的赌博:抛硬币,如果正面朝上,则赌徒赢得赌注相应的奖励;如果反面朝上,则赌徒失去赌注。这个策略不过是:每当赌徒失败,就在下一次加倍上一次的赌注使得,第一次胜利即可赢回所有的损失。假设赌徒的赌资和时间一同趋向无限大,则赌博获利的概率趋向 $1$。
 
\par 概率论中的鞅论,是由 Paul L\'{e}vy 在 1934 年中引入的,尽管当时他并未为此命名。鞅这个命名由 Ville 在 1939 年引入,同样是他将鞅的概念拓宽至连续时间的鞅。鞅论中许多的原始发展由 Joseph Leo Doob 等人所完成,其中一个知名的结果便是 Doob 极大不等式。这些工作的部分动机,就是证明必胜赌博策略的不可能性。

\par 经典鞅论的研究,或多或少的对应着调和分析的研究,主要的结果包括但不限于:Burkholder,Gundy 和 Silverstein 在文章 \cite{burkholder1971maximal} 中完成的 Hardy 空间的极大函数刻画。鞅论版本的 $H_1^\ast=BMO$ 也紧随着 Ferferman 和 Stein 在文章 \cite{fefferman1972h} 中完成的 $H_1^\ast=BMO$。归功于鞅论设定的结构简单,许多想法和方法在这个舞台上更加容易实现。此外,鞅论也有关于调和分析的应用:由 Burkholder 提出的一类重要的 Banach 空间性质(称作 UMD 性质,Unconditionality of Martingale Differences),被证明了等价于 Banach 空间上 Hilbert 变换的有界性\cite[Chapter~6]{pisier2016martingales}。鞅论的重要性,特别地体现在 Banach 值中的鞅论:20 世纪后半,Banach 空间的几何学迅速发展,其中证明可凹性与 Radon-Nikodym 性质等价时,就是通过特殊的鞅的构造\cite{刘培德2017鞅空间理论的新进展}。随着 Pisier 在 1975 年的文章 \cite{pisier1975martingales},Banach 值的鞅论被认为具有很大发展潜力。有关 Banach 值的鞅论的详细讨论,合适的材料是 \cite{pisier2016martingales}。

\par 非交换鞅论的蓬勃发展始于 1997 年 Pisier 和许全华在文章\cite{PX1} 中的讨论。关于鞅论研究的历史,或者鞅论和调和分析的关联,推荐查阅文献综述\cite{刘培德2017鞅空间理论的新进展}。

\par 鞅论是调和分析中 Fourier 级数理论的概率论类似物:在调和分析中,我们将一个可积函数对应到一个傅里叶系数序列;在鞅论中,我们将一个可积函数对应到一个鞅差序列。在调和分析中,傅里叶基是正交序列;在鞅论中,鞅差序列是正交序列。在调和分析中,我们对傅里叶系数序列用乘一个给定的有界序列做变换,再重新生成一个函数,这便是乘子理论;在鞅论中,我们对一个鞅差序列乘以一个给定的可预测序列,再重新生成一个鞅,这便是鞅变换理论。

\section{问题的研究现状}
我们按照时间顺序简单介绍非交换鞅领域的部分研究结果。
\par 最早的关于非交换鞅论的研究,是 Cuculescu 在 1971 年证明的非交换版本 Doob 极大不等式 \cite{C}。Cuculescu 在证明中给出了相应于一个鞅 $x$ 和一个正实数 $\lambda$ 的投影算子,而这个投影算子的构造可以被推广为相应于 $x$ 和 $\lambda$ 的停时。
\par 在这之后,非交换鞅论许久未有进展,直到 Pisier 和许全华在 1997 年的文章\cite{PX1},完成了非交换 Burkholder-Gundy 不等式,证明的关键点是迭代方法和对偶性。附录中还给出了非交换版本的 $H_1^\ast=BMO$。
\par 自文章\cite{PX1} 后,许多的经典鞅论不等式相关结果得到了非交换版本:2002 年,Randrianantoanina 证明了非交换鞅变换的弱 $(1,1)$ 型不等式\cite{MR1929141-2002-Narcisse}。
这一结果的新证明便是本文讨论的 Gundy 分解定理的一例应用。2005 年\cite{MR2167096-2005-Narcisse},Randrianantoanina 证明了对于 $L_1\cap L_2$ 中的有界鞅 $x=(x_n)_{n\geq1}$,存在两个鞅差序列 $a,b$ 使得 $dx_n=a_n+b_n,n\geq 1$ 满足
    \[\left\|\left(\sum_{n=1}^\infty
    a_n^*a_n\right)^{1/2}\right\|_2+\left\|\left(\sum_{n=1}^\infty
    b_nb_n^*\right)^{1/2}\right\|_2\le2\|x\|_2,\]
以及
    \[\left\|\left(\sum_{n=1}^\infty
    a_n^*a_n\right)^{1/2}\right\|_{1,\infty}+\left\|\left(\sum_{n=1}^\infty
    b_nb_n^*\right)^{1/2}\right\|_{1,\infty}\lesssim\|x\|_1.\]
这是非交换均方函数的估计。
\par 为了证明鞅不等式,一个重要的技巧是鞅的分解,例如 Gundy 分解定理。所以为了方便证明非交换鞅不等式,自然地希望有非交换 Gundy 分解定理:2006 年,Parcet 在文章 \cite{parcet2006gundy} 中证明了非交换 Gundy 分解定理并给出了一些应用。这也是本文的主要参考文献。
2007 年\cite{MR2319715-2007-Narcisse-Conditioned-square-functions},Randrianantoanina 证明了对于 $L_2$ 有界的鞅 $x=(x_n)_{1\leq n\leq N}$,存在三个适应序列 $a,b,c$ 满足:$dx_n=a_n+b_n+c_n$,$\|a\|_{L_2(\M,l^2_C)}+\|b\|_{L_2(\M,l^2_C)}+\|c\|_{L_2(\M,l^2_R)}\leq \|x\|_2$,以及如下弱 $(1,1)$ 型估计
    \[  \Biggl\|\sum_{n=1}^N a_n\otimes
    e_{n,n}\Biggr\|_{L^{1,\infty}(\M\overline{\otimes} \mathcal{B}(l^2_N))}
    +\Biggl\|\Bigl(\sum_{n=1}^N
    \E_{n-1}(|b_n|^2)\Bigr)^{1/2}\Biggr\|_{1,\infty}\\
    +\Biggl\|\Bigl(\sum_{n=1}^N
    \E_{n-1}(|c_n^*|^2)\Bigr)^{1/2}\Biggr\|_{1,\infty}\lesssim\|x\|_1.    \]
此结果的部分估计也可以使用非交换 Gundy 分解定理完成。
\par 经典的鞅论的一个重要的主题是:鞅的控制导出的鞅不等式。其中一种简单的经典鞅的控制关系是鞅的微分从属关系: $f,g$ 为两个经典鞅满足 $|df_n|\leq|dg_n|$ 对任意 $n\geq1$ 成立;这可以认为是经典鞅变换的推广。这时候可以实现均方函数的控制:$S(f)\leq S(g)$。自然地,我们希望在非交换中做出类似的结果。2020年,焦勇,Randrianantoanina,吴恋 和 周德俭 在文章\cite{MR4072235-CSU-Square;differentially-subordinate} 中完成了对于非交换弱微分从属鞅的均方函数的弱 $(1,1)$ 估计、小均方函数(也称为条件均方函数)的弱 $(1,1)$ 估计;其中条件均方函数的估计需要分解 $dy$ 为三个适应序列,而均方函数的估计将 $y$ 分解为两个鞅差序列。值得指出的是,在\cite[Theorem~3.1]{MR4072235-CSU-Square;differentially-subordinate}中,给出了相应于弱微分从属鞅的 Gundy 分解;相应于削弱的条件,结论减弱了对分解出的较好部分鞅 $\alpha$ 的有界性:从 $L_1,L_\infty$ 有界性以及 H\"{o}lder 不等式推出的 $L_2$ 有界性,变化为只有 $L_2$ 有界性。随后,在 2022 年,焦勇,Os\k{e}kowski,吴恋 和 左雅慧 在文章\cite{MR4504935-CSU-weakly-dominated-martingales} 中引入了又一种 Gundy 分解。并且依此证明了:对于 $L_2$ 中的自伴有界鞅 $x=(x_n)_{n\geq}$ 以及被 $x$ 弱控制的鞅 $y$ ,则可以将 $y$ 分解为两个鞅,即有两个鞅 $y^c,y^r$ 满足 $y=y^c+y^r$ 并且使得
    \[\|{S_C(y^c)}\|_{1,\infty}+\|{S_R(y^r)}\|_{1,\infty}\lesssim\|x\|_1.\]
\chapter{预备的定义和结果}


\newcommand{\N}{\mathbb{N}}
\newcommand{\Z}{\mathbb{Z}}
\newcommand{\R}{\mathbb{R}}
\newcommand{\C}{\mathbb{C}}


\let\cal\mathcal
\allowdisplaybreaks

\newcommand{\dem}{\noindent {\sl 证明: }}
\newcommand{\fin}{\hspace*{\fill} $\square$ \vskip0.2cm}
% \newcommand{\fin}{\pushQED{\qed}\qedhere}

%%%%%%%%%%%%%%%%%%%%%%%%%%%%%%%%%%%%%%%%%%%%%%%%%%%%%%%%%%%%%%%%%%%%

为了主线的清晰,本文将只考虑标量值的鞅以及非交换鞅;此外的重要主题即是 Banach 值的鞅。 Banach 值的鞅性质与 Banach 空间的几何性质(例如一致光滑性或者一致凸性)密不可分。
\par 我们在本章中进行预备知识的说明,这些定义和结果对该领域的专家可能是熟知的。

%%%%%%%%%%%%%%%%%%%%%%%%%%%%%%%%%%%%%%%%%%%%%%%%%%%%%%%%%%%%%%%%%%%%

\section{基础概率方面的预备内容}
    我们需要一个测度论的经典结果。这里给出一种一石二鸟的证明\cite{einsiedler2017functional}。
    \begin{theorem}[Radon-Nikodym-Lebesgue]
        设有 $(\Omega,\mathcal{F})$ 是一个可测空间,$\mu$ 是其上 $\sigma$-有限正测度,$\nu$ 是其上复测度。
        则存在唯一可积函数 $f\in L_1(\mu)$ 以及测度 $\lambda$ 使得 $\nu= f\mu+\lambda$ 并且 $\lambda\perp\mu$。
    \end{theorem}
    这里我们用 $f\mu$ 表示由测度 $\mu$ 和密度 $f$ 定义的测度。
\begin{proof}
    首先不妨假设 $\mu$ 是一个有限正测度并且 $\nu$ 也是正测度。对于一般的 $\mu$ 根据 $\sigma$-有限性,将 $\Omega$ 分解为可数个互不相交的 $\mu$-有限的子集,在这每个子集上我们就可以找出想要的函数和测度。当这些结论均得到证明,对于一般的实测度 $\nu$ 我们通过拆分正部负部可以证明同样的结果。最后对于复测度 $\nu$,我们考虑他的实部虚部分别应用对于实测度的结果,即可得到结论。
    \par 现在,我们定义一个新的测度 $m:=\mu+\nu$ 并且考虑相应的 Hilbert 空间 $L_2(m)$。在其上我们定义如下线性泛函:
        \[\phi(g):=\int gd\nu.\]
    这是有界线性泛函,因为
        \[|\phi(g)|\leq\int |g|dm\leq \|g\|_{L_2(m)}\|1\|_{L_2(m)},\]
    其中我们利用了 $m=\mu+\nu\geq\nu$ 即 $\mu$ 的正性,以及 Cauchy-Shcwarz 不等式。于是,由 Hilbert 空间上的 Riesz 表示定理\cite[Corollary~3.19]{einsiedler2017functional},有某个函数 $k\in L_2(m)$ 使得
        \begin{equation}\label{eqn:Radon-Nikodym}
            \int gd\nu=\int gkd m,\forall g\in L_2(m).
        \end{equation}
    我们断言,$k$ 是关于 $m$ 几乎处处取值于 $[0,1]$ 中的:事实上,任取 $B\in\mathcal{F}$ 我们有 $0\leq\nu(B)\leq m(B)$,所以取 $g=\mathbb{1}_B\in L_2(m)$ 可得
        \[0\leq\int_{B}kdm\leq m(B).\]
    取 $B=\{k<0\}$ 和 $B=\{k>1\}$ 可得想要的结果。 
    \par 由于 $m=\mu+\nu$,我们可以将 \eqref{eqn:Radon-Nikodym} 重新写为
        \begin{equation}\label{eqn:rewrite}
            \int g(1-k)d\nu=\int gkd\mu.
        \end{equation}    
    这是因为,$k$的选取保证这对于简单函数 $g$ 是合理的,因此由单调收敛定理对于非负可测函数也是成立的。现在定义测度的奇异部分支撑 $\Omega_{s}:=\{k=1\}$ 和测度的绝对连续支撑 $\Omega_{\mu}:=\Omega\setminus\Omega_{s}=\{k<1\}$,并且设 $\nu_s=\left.\nu\right|_{\Omega_s}$。
    按照定义,有 $\nu_s(\Omega_s)=0$ 而且按照式子 \eqref{eqn:Radon-Nikodym} 取 $g=\mathbb{1}_{\Omega_s}$ 有 $\mu(\Omega_s)=0$。以上说明了 $\nu_s\perp\mu$,即测度相互正交。我们再定义测度 $\nu$ 的绝对连续部分 $\nu_{a}:=\left.\nu\right|_{\Omega_\mu}$,它满足了 $\nu=\nu_s+\nu_a$。最后,命函数
        \[f:=\frac{k}{1-k}\mathbb{1}_{\Omega_\mu}.\]
    则对于任意可测函数 $g\geq0$ 我们使用 \eqref{eqn:rewrite} 得到
        \[\int_{\Omega} gd\nu_a=\int_{\Omega_\mu}\frac{g}{1-k}(1-k)d\nu=\int_{\Omega_\mu}\frac{g}{1-k}kd\mu=\int_{\Omega}fd\mu.\]
    这表明了 $f=\frac{d\nu_{a}}{d\mu}$ 是 Radon-Nikodym 微商,而且 $\nu_{a}\ll \mu$。根据 $\nu_a(\Omega)$ 有限,$f$ 是 Radon-Nikodym 微商以及 $f\geq0$ 可知 $f\in L_1(\mu)$。
    \par 对于唯一性,我们假设 $f_1,f_2$ 都是 $\nu_a$ 关于 $\mu$ 的 Radon-Nikodym 微商,则
        \[\int_B f_1d\mu=\nu_{a}(B)=\int_B f_2d\mu,\]
    对于所有 $B\in\mathcal{F}$ 成立。考察 $B:=\{f_1>f_2\}$ 和 $B:=\{f_2>f_1\}$ 即可知道 $f_1=f_2,\mu$-几乎处处成立。
\end{proof}

    定理的其他证明可以参考\cite*{david2018probability,folland1999real},方法可谓各有千秋。
    当然,上述定理在鞅论中,重要的是应用于:$\mu$ 是概率测度,而 $\nu=\left.f\mu\right|_{\mathcal{F_n}}$,
    对任意 $f\in L_1(\mu)$ 以及任意 $n\geq0$。这样就可以对每个 $f\in L_1(\nu)$ 定义出一个鞅,定义见下文。
    此外,这样就给出条件期望算子 $\E_n:=\E(\ |\mathcal{F}_n)$ 的定义。条件期望算子有着丰富的性质,包括但不限于 线性、有界性及 $L_0(\mathcal{F}_n)$-模性质。
    这些性质也是我们在后续的非交换框架下关注的。
    \par 考虑一个测度空间 $(\Omega,\mathcal{F},\mu)$。
    其上的一列 $\sigma$ 代数流是指一列 $\sigma$ 代数 $(\mathcal{A}_n)_{n\geq0}$ 满足 $\mathcal{A_n}\subseteq \mathcal{A_{n+1}}$。
    如同我们对测度空间所称 $\sigma$-有限的,我们称这个 $\sigma$ 代数流是 $\sigma$-有限的,如果对每个 $n\geq0$ 都有 $\mu$ 在 $\mathcal{F}_n$ 上有限。
    
    \par 现在,我们定义:一个函数序列 $f=(f_n)_{n\geq0}$ 称为适应于代数流 $(\mathcal{A}_n)_{n\geq0}$ 的,如果他满足:
    $f_n\in L_0(\Omega,\mathcal{F}_n)$ 对于所有的 $n\geq0$ 成立。
    \par 最后,一个关于 $\sigma$-代数流 $(\mathcal{A}_n)_{n\geq0}$ 的鞅, 是一个适应于代数流 $(\mathcal{A}_n)_{n\geq0}$ 的函数序列 $(f_n)_{n\geq0}$ 使得 $\E_{n-1} f_n=f_{n-1}$。
%%%%%%%%%%%%%%%%%%%%%%%%%%%%%%%%%%%%%%%%%%%%%%%%%%%%%%%%%%%%%%%%%%%%

\section{非交换鞅方面的预备内容}
    一个 Banach 代数指的是一个 Banach 空间 $\mathcal{A}$,其上有(收缩的)乘法运算,即一个(收缩的)双线性算子。
    一个 $C^\ast$-代数是指一个 Banach 代数 $\mathcal{B}$,其上配备有一个对合算子,满足:幂等性,共轭线性性,以及 $C^\ast$-范数限制:$\|xx^\ast\|=\|x\|^2$。
    按照 GNS 表示,我们总是可以假设 $\mathcal{B}$ 是 $\mathcal{B}(H)$ 的子代数,其中 $H$ 是某个 Hilbert 空间。
    对于一个 Hilbert 空间 $H$ 上的有界算子空间 $\mathcal{B}(H)$,我们会考虑他的弱算子拓扑(WOT),定义为如下半范数族导出的极限拓扑:$\{p_{\xi,\eta}|\xi\in H,\eta\in H^\ast\}$,其中 $p_{\xi,\eta}(x)=|\langle x\xi,\eta\rangle|$。一个 VNA 指的是一个在 $\mathcal{B}(H)$ 中的 WOT-闭单位子代数。
%%%%%%%%%%%%%%%%%%%%%%%%%%%%%%%%%%%%%%%%%%%%%%%%%%%%%%%%%%%%%%%%%%%%

% \chapter{预备的定义和结果}
\section{非交换对称空间}
本节介绍一些预备定义和结果,这些定义和结果对该领域的专家中可能是熟知的,
并且在本文的其余部分中将会用到。在全文中,$\M$是一个半有限 von Neumann 代数,
配备正规的忠实的半有限迹$\T$。$\M$的单位元用$\mathbf{1}$表示。
对于$0 < p \leq \infty$,$L_p(\M,\tau)$是相应的非交换$L_p$空间,参见例如\cite{N,PX2}。
注意,如果$p=\infty$,$L_\infty(\M,\tau)$就是带有通常算子范数的$\M$;
还要回忆一下,对于$0< p<\infty$,$L_p(\M,\tau)$上的(拟)范数由以下方式定义:
\[\|x\|_p = \left(\T(|x|^p)\right)^{1/p}, \quad \text{其中} \quad |x|=(x^*x)^{1/2}.\]
% \subsection{非交换对称空间}

假设$\M$作用于Hilbert空间$H$上。对于H上的闭稠密定义的算子$x$,
如果$x$与$\M$的交换子$\M'$中的每个酉算子$u$都可交换,则称$x$与$\M$是"附属"的。

设$a$是$H$上的一个闭稠密自伴算子,其谱分解为$a = \int_{\R} s d e^a_s$,
其中$s$是谱参数,$e^a_s$是与$a$相关的谱测度。对于任意的Borel子集$B \subseteq \R$,
我们用$\chi_B(a)$表示相应的谱投影$\int_{\R} \chi_B(s) d e^a_s$。
如果对于一个$\M$相关的算子$x$,存在$s > 0$使得$$\tau \big( \chi_{(s,\infty)} (|x|) \big) < \infty$$
则称其为"$\tau$-可测"。

$\tau$-可测算子$x$的"广义奇异值" $\mu(x): \R_+ \to \R_+$定义为
$$\mu_t(x) = \inf \Big\{ s > 0 \, \big| \ \tau \big( \chi_{(s,\infty)}(|x|) \big) \le t \Big\}.$$
对函数$\mu(x)$的详细阐述,请参考文献\cite{FK}。对于区间$(0,\tau(\mathbf{1}))$上的一个重排不变
拟Banach函数空间$E$,我们定义\emph{非交换对称空间}$E(\mathcal{M},\tau)$如下:
$$E(\mathcal{M},\tau) := \Big\{x \in L_0(\mathcal{M},\tau) \,
 \big| \ \mu(x) \in E \Big\}, \quad \text{且} \quad \left\| x \right\|_{E(\M,\tau)} := \left\|\mu(x)\right\|_E$$
其中$L_0(\mathcal{M},\tau)$表示$\tau$-可测算子的$*$-代数。
已知当$E$是Banach(或拟Banach)函数空间时,$(E(\mathcal{M},\tau), \|\cdot\|_{E(\M,\tau)})$
是Banach(或拟Banach)空间。关于对这个构造更深入的讨论,我们建议读者参考文献\cite{DDdP,X2}。
对于$E$是Banach空间的情况,有以下包含关系:
\[ L_1(\M,\tau) \cap \M \subseteq E(\M,\tau) \subseteq L_1(\M,\tau) + \M \]
这里的包含映射的范数均为1(上述$L_1(\M,\tau) \cap \M$与$L_1(\M,\tau) + \M$中的范数是交与和的通常范数)。$E(\M,\tau)$的\emph{K\"othe对偶} $E(\M,\tau)^\times$定义为所有满足对于所有$y \in E(\M,\tau)$,都有$xy \in L_1(\M,\tau)$的$x \in L_0({\M},\tau)$的集合。通过定义范数为:
$$\left\|x\right\|_{E(\M,\tau)^\times} := \sup \Big\{ \T(|xy|) \, \big| \ y \in E(\M,\tau), \, \|y\|_{E(\M,\tau)} \leq 1 \Big\}$$
K\"othe对偶$E(\M,\tau)^\times$是一个Banach空间。关于交换情况下K\"othe对偶的基本性质,可以在文献\cite{LT}中找到。对于非交换情况,建议读者参考\cite{DDP3}。我们注意到从\cite{DDP3}中得知,如果$E$是一个重排不变函数空间,那么$(E(\M,\tau)^\times, \|\cdot\|_{E(\M,\tau)^\times})$可以与空间$(E^\times(\M,\tau), \|\cdot\|_{E^\times(\M,\tau)})$等同。
特别地,我们有以下结果:
\begin{eqnarray*}
(L_1(\M,\tau) +\M)^\times & = & L_1(\M,\tau) \cap \M, \\
(L_1(\M,\tau)\cap \M)^\times & = & L_1(\M,\tau) +\M.
\end{eqnarray*}

非交换空间中的相对弱紧性在本文中起着重要作用。
下面,我们明确陈述我们在接下来的章节中所需的一个特性。
首先,我们定义$S_0(\M,\tau) :={\M}_0 \cap (L_1(\M,\tau) +\M)$,其中
\[ {\M}_0 := \Big\{ x \in L_0({\M},\tau) \, \big| \ \mu_t(x)\to 0 \ \text{as} \ t \to \infty \Big\}. \]

\begin{theorem} \label{weakcom2} {\rm\cite[Theorem~5.4]{DSS}.}
假设对称空间$E(\M,\tau)$包含在$S_0(\M,\tau)$中,
并且$K$是$E(\M,\tau)^\times$的有界子集。那么,以下陈述等价:
\begin{itemize}
\item[(i)] $\mu(K)$在相对于$\sigma(E^\times, E)$拓扑下是紧的;
\item[(ii)] $K$在相对于$\sigma(E(\M,\tau)^\times, E(\M,\tau))$拓扑下是紧的。
\end{itemize}
\end{theorem}

本文主要关注非交换$L_p$空间和\emph{非交换弱$L_1$空间}。在可测算子的对称空间构造之后,
非交换弱$L_1$空间$L_{1,\infty}(\mathcal{M}, \T)$被定义为
满足以下拟范数有界的$x \in L_0(\mathcal{M},\tau)$的集合:
\[ \left\|x\right\|_{1,\infty} = \sup_{t > 0} t \mu_t(x) = \sup_{\lambda > 0} \lambda \tau \big( \chi_{(\lambda, \infty)} (|x|) \big) < \infty. \]
类似于交换情况,可以轻松验证如果$x_1, x_2 \in L_{1,\infty}(\M,\tau)$,
则 $\|x_1 +x_2\|_{1,\infty} \leq 2\|x_1\|_{1,\infty} +2\|x_2\|_{1,\infty}$。
事实上,在接下来的推导中我们会多次使用以下更一般的拟三角不等式。
其简短证明可以在 \cite[Lemma~1.2]{R3}中找到。

\begin{lemma} \label{Quasi-Triangle}
对于$L_{1,\infty}(\mathcal{M},\tau)$中的两个算子$x_1, x_2$和$\lambda > 0$,我们有
$$\lambda \, \tau \Big( \chi_{(\lambda,\infty)} \big( |x_1+x_2| \big) \Big) \le 2 \lambda \, \tau \Big( \chi_{(\lambda/2, \infty)} \big( |x_1| \big) \Big) + 2 \lambda \, \tau \Big( \chi_{(\lambda/2, \infty)} \big( |x_2| \big) \Big).$$
\end{lemma}

设 $\mathsf{P} = \{ p_i \}_{i=1}^m$ 是 $\mathcal{M}$ 中一组两两正交的有限投影。我们考虑相对于 $\mathsf{P}$ 的{\it 三角截断},它是定义在 $L_0(\mathcal{M},\tau)$ 上的映射,具体定义如下:
\[ \mathcal{T}^{(\mathsf{P})} x = \sum_{i=1}^m \sum_{i\leq j} p_i x p_j. \]
接下来的引理将在后续中使用。

\begin{lemma} \label{Truncation} {\rm\cite[Proposition~1.4]{R3}} 存在一个绝对常数 $\mathrm{c}>0$,使得对于任意一族有限两两正交投影序列 $(\mathsf{P}_k)_{k\geq 1}$ 和任意序列 $(x_k)_{k\geq 1}$,其中 $x_k \in L_1(\mathcal{M},\tau)$,有
\[
 \Big\| \Big( \sum_{k\geq 1} \big| \mathcal{T}^{(\mathsf{P}_k)} x_k \big|^2 \Big)^{1/2} \Big\|_{1,\infty} \le \mathrm{c} \sum_{k\geq 1} \|x_k\|_1.
\]
\end{lemma}

\section{非交换鞅}
考虑$\mathcal{M}$的von Neumann子代数$\mathcal{N}$(即$\mathcal{M}$的一个弱$*$闭$*$-子代数)。
从$\mathcal{M}$到$\mathcal{N}$的\emph{条件期望}$\E:\mathcal{M}\to\mathcal{N}$是一个正的、收缩的投影算子。
如果伴随映射$\E^*$满足$\E^*(\mathcal{M}_*)\subset\mathcal{N}_*$,
则条件期望$\E$被称为\emph{正规的}。在这种情况下,存在一个映射$\E_*:\mathcal{M}_*\rightarrow\mathcal{N}_*$,其伴随映射是$\E$。
请注意,这样的正规条件期望存在当且仅当将$\tau$限制在von Neumann子代数$\mathcal{N}$上仍然是半有限的(例如,参见\cite[Theorem~3.4]{T})。
任何这样的条件期望都是保迹的(即,$\tau\circ\E=\tau$),并且满足双模性质:
\[\E(axb)=a\E(x)b\quad\forall a,b\in\mathcal{N}\ \forall x\in\mathcal{M}.\]

设$(\mathcal{M}_n)_{n\ge 1}$是$\mathcal{M}$的递增von Neumann子代数序列,使得这些$\mathcal{M}_n$的并在弱$^*$拓扑下在$\mathcal{M}$中稠密。对于每个$n\geq 1$,假设存在一个正规条件期望$\E_n:\mathcal{M}\to\mathcal{M}_n$。
请注意,对于每个$1\leq p<\infty$和$n\ge 1$,$\E_n$可以扩展为一个正的收缩算子$\E_n:L_p(\mathcal{M},\tau)\to L_p(\mathcal{M}_n,\tau|_{\M_n})$。相对于子代数流$(\mathcal{M}_n)_{n\ge 1}$的\emph{非交换鞅}是一个序列$x=(x_n)_{n\ge 1}$,其中$x_n\in L_1(\mathcal{M},\tau)$,满足以下条件:
\[
\E_m(x_n)=x_m\quad\forall 1\leq m\leq n<\infty.
\]
如果此外对于某个$1\leq p\leq \infty$,$x\subset L_p(\mathcal{M},\tau)$,则称$x$为一个\emph{$L_p$-鞅}。在这种情况下,我们定义
\[\left\|x\right\|_p:=\sup_{n\ge 1}\left\|x_n\right\|_p.\]
如果$\|x\|_p<\infty$,则称$x$为一个\emph{$L_p$-有界鞅}。给定一个鞅$x=(x_n)_{n\ge 1}$,我们约定$x_0=0$。然后,与$x$相关的鞅差序列$dx=(dx_k)_{k\ge 1}$由以下定义:
$$dx_k=x_k-x_{k-1}.$$

我们现在描述非交换鞅的平方函数。
参考Pisier和许全华老师的研究\cite{PX1},我们考虑以下的行和列版本的平方函数。
给定一个鞅差序列 $dx = (dx_k)_{k \ge 1}$ 和 $n \geq 1$,我们定义 $x$ 的
\emph{行平方函数}和\emph{列平方函数}如下:
\[
\cal{S}_{C,n} (x) := \Big(\sum^n_{k=1}|dx_{k}|^{2} \Big)^{{1}/{2}}
\quad \text{和} \quad \cal{S}_{R,n}(x) :=
\Big(\sum^n_{k=1}|dx_{k}^*|^{2} \Big)^{{1}/{2}}.
\]
让我们考虑一个在区间 $[0, \T({\bf 1}))$ 上具有重新排列不变性的(拟)Banach函数空间 $E$。
然后,我们定义空间 $E(\mathcal{M}, \tau; l^2_C)$ 和 $E(\mathcal{M}, \tau; l^2_R)$ 
为有限序列 $a = (a_k)_{k \ge 1}$ 在 $E(\mathcal{M}, \tau)$ 中的向量空间关于以下范数的完备化:
\begin{eqnarray*}
\|a\|_{E(\M,\tau;l^{2}_{C})} & = & \Big\| \Big( \sum_{k \geq 1}
|a_{k}|^{2} \Big)^{{1}/{2}} \Big\|_{E(\M,\tau)}, \\ \|a\|_{E(\M,
\tau; l^{2}_{R})} & = & \Big\| \Big( \sum_{k \geq 1}
|a_{k}^{*}|^{2} \Big)^{{1}/{2}} \Big\|_{E(\M,\tau)}.
\end{eqnarray*}

如果鞅差序列 $dx$ 属于 $E(\M,\tau;l^{2}_{C})$(或 $E(\M,\tau; l^{2}_{R})$),当且仅当序列 $(\mathcal{S}_{C,n}(x))_{n \ge 1}$
(或 $(\mathcal{S}_{R,n}(x))_{n \ge 1}$)在 $E(\M,\tau)$ 中有界。在这种情况下,极限
\[
\cal{S}_{C}(x) := \Big( \sum^{\infty}_{k=1} |dx_{k}|^{2}
\Big)^{{1}/{2}} \quad \mbox{和} \quad \mathcal{S}_{R}(x) := \Big(
\sum^{\infty}_{k=1}|dx_{k}^{*}|^{2} \Big)^{{1}/{2}} \]
是 $E(\M,\tau)$ 中的元素。这两个版本的平方函数在接下来的部分非常关键。

下面关于正鞅的结果已经广为人知,可以参考\cite{C},并且可以视作Doob极大函数的经典弱$(1,1)$型有界性
在非交换下的类比推广。下面给出的自伴鞅的推广在接下来的部分中起着关键作用。
其证明是Cuculescu的原本论证上进行微小调整,但为了完整性,我们将提供这些细节。

\begin{proposition}\label{Cuculescu}
如果$x=(x_{n})_{n\ge 1}$是一个自伴的$L_1$-有界鞅,并且$\lambda$是一个正实数,则存在一个在von Neumann代数$\M$中的递减投影序列
$$q_0^{(\lambda)} \geq q_1^{(\lambda)} \geq q_2^{(\lambda)} \geq \ldots$$
满足以下性质:
\begin{itemize}
\item[(i)] 对于每个$n\ge 1$,$q_{n}^{(\lambda)}\in \M_{n}$;
\item[(ii)] 对于每个$n\ge 1$,$q_{n}^{(\lambda)}$与$q_{n-1}^{(\lambda)}x_{n}q_{n-1}^{(\lambda)}$交换;
\item[(iii)] 对于每个$n\ge 1$,$|q_{n}^{(\lambda)}x_{n}q_{n}^{(\lambda)}|\leq \lambda q_{n}^{(\lambda)}$;
\item[(iv)] 如果我们设$q^{(\lambda)}=\bigwedge_{n=1}^{\infty}q_{n}^{(\lambda)}$,则$$\tau\left({\bf 1}-q^{(\lambda)}\right)\leq \frac{1}{\lambda}\|x\|_1.$$
\end{itemize}\end{proposition}

\begin{proof}
让 $q_{0}^{(\lambda)} = {\bf 1}$,并对 $n \geq 1$ 进行如下的归纳定义:
%\begin{equation*}
\[
q_{n}^{(\lambda)} := q_{n-1}^{(\lambda)} \chi_{[-\lambda,\lambda]}
\Big( q_{n-1}^{(\lambda)} x_{n} q_{n-1}^{(\lambda)} \Big) =
\chi_{[-\lambda,\lambda]} \Big( q_{n-1}^{(\lambda)} x_{n}
q_{n-1}^{(\lambda)} \Big) q_{n-1}^{(\lambda)}.\]
%\end{equation*}
上述等式成立,因为 $q_{n-1}^{(\lambda)}$ 与 $q_{n-1}^{(\lambda)} x_{n} q_{n-1}^{(\lambda)}$ 交换,这是由于 $q_{n-1}^{(\lambda)}$ 是一个投影。这显然给出了一个递减的投影序列。根据归纳法,条件 (i) 成立。此外,条件 (ii) 直接来自上述定义。对于条件 (iii),注意对于每个 $n\geq 1$,有:
\begin{eqnarray*}
q_{n}^{(\lambda)} x_{n} q_{n}^{(\lambda)} & = & q_{n}^{(\lambda)}
(q_{n-1}^{(\lambda)} x_{n} q_{n-1}^{(\lambda)}) q_{n}^{(\lambda)}
\\ & = & q_{n-1}^{(\lambda)} \chi_{[-\lambda,\lambda]}
(q_{n-1}^{(\lambda)} x_{n} q_{n-1}^{(\lambda)})
q_{n-1}^{(\lambda)} x_{n} q_{n-1}^{(\lambda)}
\chi_{[-\lambda,\lambda]} (q_{n-1}^{(\lambda)} x_{n}
q_{n-1}^{(\lambda)}) q_{n-1}^{(\lambda)}.
\end{eqnarray*}
因此,$-\lambda q_{n}^{(\lambda)}\leq q_{n}^{(\lambda)} x_{n}
q_{n}^{(\lambda)}\leq \lambda q_{n}^{(\lambda)}$,从而满足条件 (iii)。为了证明条件 (iv),我们使用 Krickeberg 分解的非交换推广版本\cite{C},依此我们可以将 $x_n= w_n - z_n$ 写成正鞅 $w = (w_n){n \ge 1}$ 和 $z = (z_n){n \ge 1}$ 的形式,其中
$$\|x\|_1=\tau(w_1 + z_1).$$ 
对于每个 $n\geq 1$ ,成立
\begin{eqnarray*}
\|x\|_1 & = & \tau \big( (w_n + z_n) q_{n}^{(\lambda)} \big) +
\sum^n_{k=1} \tau \Big( (w_n + z_n) (q_{k-1}^{(\lambda)} -
q_{k}^{(\lambda)}) \Big)
\\ & = & \tau \big( q_{n}^{(\lambda)} (w_n + z_n)
q_n^{(\lambda)} \big) + \sum^n_{k=1} \tau \Big( \E_k (w_n +
z_n) (q_{k-1}^{(\lambda)} - q_{k}^{(\lambda)}) \Big).
\end{eqnarray*}
由于 $\tau \big( q_{n}^{(\lambda)} (w_n + z_n) q_{n}^{(\lambda)})
\geq 0$,我们有
\begin{eqnarray*}
\|x\|_1 & \geq & \sum^{n}_{k=1} \tau \Big( (q_{k-1}^{(\lambda)} -
 q_{k}^{(\lambda)}) (w_k + z_k) (q_{k-1}^{(\lambda)} -
 q_{k}^{(\lambda)}) \Big) \\ & \geq & \tau \Big( \sum^{n}_{k=1}
 \Big| (q_{k-1}^{(\lambda)} - q_{k}^{(\lambda)}) (w_k - z_k)
 (q_{k-1}^{(\lambda)} - q_{k}^{(\lambda)}) \Big| \Big)
\\ & = & \tau \Big( \sum^{n}_{k=1} \Big| (q_{k-1}^{(\lambda)} -
q_{k}^{(\lambda)}) (q_{k-1}^{(\lambda)} x_k q_{k-1}^{(\lambda)})
(q_{k-1}^{(\lambda)} - q_{k}^{(\lambda)}) \Big| \Big).
\end{eqnarray*}
根据 $q_{k}^{(\lambda)}$ 的定义,容易得到
\begin{eqnarray*}
q_{k-1}^{(\lambda)} - q_{k}^{(\lambda)} & = & q_{k-1}^{(\lambda)}
\Big( \chi_{(-\infty, -\lambda)} (q_{k-1}^{(\lambda)} x_{k}
q_{k-1}^{(\lambda)}) + \chi_{(\lambda, \infty)}
(q_{k-1}^{(\lambda)} x_{k} q_{k-1}^{(\lambda)}) \Big) \\ & = &
\Big( \chi_{(-\infty, -\lambda)} (q_{k-1}^{(\lambda)} x_{k}
q_{k-1}^{(\lambda)}) + \chi_{(\lambda, \infty)}
(q_{k-1}^{(\lambda)} x_{k} q_{k-1}^{(\lambda)}) \Big)
q_{k-1}^{(\lambda)}.
\end{eqnarray*}
于是,如果 $q_{k-1}^{(\lambda)}
x_{k}q_{k-1}^{(\lambda)}=\int_{\mathbb R} t de_t^{(k)}$ 是 $q_{k-1}^{(\lambda)} x_{k}q_{k-1}^{(\lambda)}$ 的谱分解,
我们得到
\begin{eqnarray*}
\lambda (q_{k-1}^{(\lambda)} - q_{k}^{(\lambda)}) & \le &
\int_{-\infty}^{- \lambda} |t| \, de_t^{(k)} +
\int_{\lambda}^{\infty} |t| \, de_t^{(k)} \\ & = & \Big|
(q_{k-1}^{(\lambda)} - q_{k}^{(\lambda)}) (q_{k-1}^{(\lambda)} x_k
q_{k-1}^{(\lambda)}) (q_{k-1}^{(\lambda)} - q_{k}^{(\lambda)})
\Big|.
\end{eqnarray*}
现在我们可以得出 $$\tau \Big( {\bf 1}-q_{n}^{(\lambda)}
\Big) \le \frac{1}{\lambda} \|x\|_1.$$ 取极限 $n \to
\infty$, 我们得到 (iv)。这完成了证明。 
\end{proof}

在接下来的内容中,我们将把命题 \ref{Cuculescu} 中的投影序列称为关于(自伴)鞅 $x$ 和(正)参数 $\lambda$ 的 Cuculescu 投影序列。在下一个结果中,我们总结了这个序列的一些基本性质,这些性质非常有用,将在下一节中出现。

\begin{proposition} \label{Randri}
设 $x = (x_n)_{n \ge 1}$ 是一个自伴 $L_1$-有界鞅,$\lambda$ 是一个正实数。那么对于每个 $n\ge 1$ ,关于 $x$ 和 $\lambda$ 的 Cuculescu 投影序列满足以下估计:

\begin{eqnarray}
\label{Est1} \sum_{k=1}^{n} \big\| q_{k-1}^{(\lambda)} x_k
q_{k-1}^{(\lambda)} - q_k^{(\lambda)} x_k q_k^{(\lambda)} \big\|_1
& \le & \|x\|_1, \\ \label{Est2} \sum_{k=1}^{n} \big\|
q_{k-1}^{(\lambda)} x_{k-1} q_{k-1}^{(\lambda)} - q_k^{(\lambda)}
x_{k-1} q_k^{(\lambda)}\big\|_1 & \le & 2 \|x\|_1, \\ \label{Est3}
\sum_{k=1}^{n} \big\| q_{k-1}^{(\lambda)} dx_k q_{k-1}^{(\lambda)}
- q_k^{(\lambda)} dx_k q_k^{(\lambda)} \big\|_1 & \le & 3 \|x\|_1.
\end{eqnarray}
此外,还有以下恒等式成立:
\begin{equation} \label{Ident1}
\sum_{k=1}^{n} q_{k-1}^{(\lambda)} dx_k  q_{k-1}^{(\lambda)}=
q_n^{(\lambda)}  x_n  q_n^{(\lambda)}+ \sum_{k=1}^{n}
(q_{k-1}^{(\lambda)}-q_{k}^{(\lambda)})x_k
(q_{k-1}^{(\lambda)}-q_{k}^{(\lambda)}).
\end{equation}
特别地,我们得到
\begin{equation} \label{Est4}
\Big\| \sum_{k=1}^{n} q_{k-1}^{(\lambda)} dx_k q_{k-1}^{(\lambda)}
\Big\|_1 \leq 2 \|x\|_1.
\end{equation}
\end{proposition}

\begin{proof}
我们将用 $(q_n)_{n \ge 0}$ 表示 $(q_n^{(\lambda)})_{n \ge
0}$,用 $q$ 表示 $q^{(\lambda)}$ (参见命题 \ref{Cuculescu})。
设 $v_k = q_{k-1} x_k q_{k-1} - q_k x_k q_k$. 由于 $q_k$
与 $q_{k-1} x_k q_{k-1}$ 交换,我们有 $v_k = (q_{k-1} -
q_k) x_k (q_{k-1} - q_k)$ 并且因此对于任意给定 $n \ge 1$,
\begin{eqnarray*}
\sum_{k=1}^n \|v_k\|_1 & = & \sum_{k=1}^n \big\| (q_{k-1} - q_k)
x_k (q_{k-1} - q_k) \big\|_1 \\ & = & \sum_{k=1}^n \big\|
\E_k \big( (q_{k-1} - q_k) x_n (q_{k-1} - q_k) \big)
\big\|_1
 \\ & \leq & \sum_{k=1}^n \big\| (q_{k-1} - q_k) x_n (q_{k-1}
- q_k) \big\|_1 \le \|x_n\|_1.
\end{eqnarray*}
因此,不等式~\eqref{Est1} 成立。对于不等式~\eqref{Est2},令 $$\sigma_k = q_k x_{k-1} q_k - q_{k-1} x_{k-1}
q_{k-1}.$$ 然后我们有,
\begin{eqnarray*}
\sigma_k & = & q_k x_{k-1} (q_k - q_{k-1}) + (q_k -q_{k-1})
x_{k-1} q_{k-1} \\ & = & q_k q_{k-1} x_{k-1} q_{k-1} (q_k -
q_{k-1}) + (q_k - q_{k-1}) q_{k-1} x_{k-1} q_{k-1}.
\end{eqnarray*}
根据 H\"{o}lder 不等式以及命题~\ref{Cuculescu},我们得出
\[\sum_{k=1}^\infty\|\sigma_k\|_1 \le 2 \sum_{k=1}^\infty
\tau(q_{k-1} - q_k) \big\| q_{k-1} x_{k-1} q_{k-1}\big\|_{\infty}
\le 2 \lambda \tau (\mathbf{1} - q)\leq 2\|x\|_1,
 \]
这证明了~\eqref{Est2}。不等式~\eqref{Est3}
直接由~\eqref{Est1} 、~\eqref{Est2} 和三角不等式得出。等式~\eqref{Ident1} 由分部求和方法得出。事实上,对于 $n\geq 1$ 我们有
\begin{eqnarray*}
\sum_{k=1}^n q_{k-1} dx_k q_{k-1} & = & \sum_{k=1}^n \big( q_{k-1}
x_k q_{k-1} - q_{k-1} x_{k-1} q_{k-1} \big) \\ & = &
\sum_{k=1}^{n-1} \big( q_{k-1} x_k q_{k-1} - q_k x_k q_k \big) +
q_{n-1} x_n q_{n-1} \\ & = & \sum_{k=1}^{n} \big( q_{k-1} x_k
q_{k-1} - q_k x_k q_k \big) + q_{n} x_n q_{n}
\end{eqnarray*}
最后,不等式~\eqref{Est4} 从~\eqref{Est1} 和~\eqref{Ident1} 得出。证明完毕。
\end{proof}

\section{非交换 Gundy 分解}
\label{Section2}

在本节中,我们将介绍 Gundy 分解定理在非交换下的推广,这也是本文的主要结果。所有适应
序列和鞅都被理解为相对于一个固定子代数流 $(\mathcal{M}_n)_{n \ge 1}$。
为了方便起见,我们假设 $\E_0=\E_1$。

\begin{theorem}\label{Main} 如果 $x=(x_n)_{n \ge 1}$ 是一个 $L_1$ 有界的鞅
且 $\lambda$ 是一个正实数时,那么就存在四个鞅 $\alpha$, $\beta$, $\gamma$、
和 $\upsilon$ 满足以下属性,对于某个绝对常数 $\mathrm{c}$:
\begin{itemize}
\item[(i)] $x=\alpha +\beta + \gamma + \upsilon$;
\item[(ii)] 鞅 $\alpha$ 满足 $$\|\alpha\|_1 \leq
\mathrm{c} \|x\|_1, \quad \|\alpha\|_2^2 \leq \mathrm{c}
\lambda\|x\|_1, \quad \|\alpha\|_\infty \leq \mathrm{c} \lambda;$$
\item[(iii)] 鞅 $\beta$ 满足 $$\sum_{k=1}^{\infty}
\|d\beta_k\|_1 \le \mathrm{c} \|x\|_1;$$
\item[(iv)] $\gamma$ 和 $\upsilon$ 是 $L_1$-鞅且
$$\max \Big\{ \lambda \tau \Big( \bigvee_{k \ge 1} \mathrm{supp}
|d\gamma_k| \Big), \, \lambda \tau \Big( \bigvee_{k \ge 1}
\mathrm{supp} \, |d\upsilon_k^*| \Big) \Big\} \le \mathrm{c}
\|x\|_1.$$
\end{itemize}
\end{theorem}

\dem 不失一般性,我们可以假设并且假设鞅$x$是正的。记 $(q_n)_{n \ge 0}$ 为与鞅 $x$ 和固定的 $\lambda>0$ 相关的 Cuculescu 投影序列。构造分为两个步骤进行。

% \vskip5pt

\noindent{\bf 第一步。}我们考虑下面的鞅差序列
%\begin{equation}\label{y}
$$dy_k := q_k dx_k q_k - \E_{k-1} (q_k dx_k q_k ) \quad \text{对于}
\quad k \geq 1.$$
%\end{equation}
很明显,$(dy_k)_{k \ge 1}$ 是一个鞅差序列,对应的鞅 $y = (y_n)_{n \ge 1}$ 是一个自伴 $L_1$ 鞅。下面的中间引理对我们的构造是至关重要的。
\begin{lemma}\label{marting-y}
鞅 $y$ 是 $L_1$ 有界的,满足 $\|y\|_1\leq  9 \|x\|_1$。
\end{lemma}
\dem 这个引理的证明主要基于命题~\ref{Randri}。事实上,对于每个$n\geq 1$,
\begin{eqnarray*}
\left\|y_n\right\|_1 & = & \Big\| \sum_{k=1}^n q_k dx_k q_k -
\E_{k-1} (q_k dx_k q_k ) \Big\|_1 \\ & \leq & \Big\|\sum_{k=1}^n
q_{k-1} dx_k q_{k-1} \Big\|_1 \\ & + & \sum_{k=1}^n \big\| q_{k-1}
dx_k q_{k-1}-q_k dx_k q_k \big\|_1  \\ & + & \sum_{k=2}^n \big\|
\E_{k-1} (q_k dx_k q_k) \big\|_1 + \big\| q_1 x_1 q_1 \big\|_1.
\end{eqnarray*}
由于对于 $2\leq k\leq n$,$$\E_{k-1} \big( q_k dx_k q_k \big) = \E_{k-1} \big( q_k dx_k q_k - q_{k-1} dx_k q_{k-1} \big),$$
该断言由估计式~\eqref{Est3}~和~\eqref{Est4}~成立。因此,引理得证。

\noindent{\bf 第二步。}设 $(\pi_n)_{n \ge 0}$ 表示相对于(自伴)鞅 $y$ 和上述固定参数 $\lambda$ 的Cuculescu投影序列。我们如下定义鞅 $\alpha$、$\beta$、$\gamma$和$\upsilon$:

\begin{equation*}\label{decomposition}
\begin{cases}
d\alpha_k &:= \pi_{k-1} \big[ q_k dx_k q_k - \E_{k-1} (q_k dx_k
q_k) \big] \pi_{k-1}, \\ d\beta_k &:= \pi_{k-1} \big[ q_{k-1}dx_k
q_{k-1}- q_k dx_k q_k + \E_{k-1} (q_k dx_k q_k) \big]\pi_{k-1}, \\
d\gamma_k &:=dx_k -  dx_k q_{k-1}\pi_{k-1},\\ d\upsilon_k &:=dx_k
q_{k-1}\pi_{k-1}-\pi_{k-1}q_{k-1}dx_k q_{k-1}\pi_{k-1}.
\end{cases} \tag{$\mathbf{G}_{\lambda}$}
\end{equation*}
显然, $d\alpha$, $d\beta$, $d\gamma$ 和 $d\upsilon$ 是鞅差序列并且 $x=\alpha +\beta +\gamma+\upsilon$。

\begin{lemma}\label{alpha-norm} 鞅 $\alpha$ 满足
$$\|\alpha\|_1 \leq 18 \|x\|_1, \quad \|\alpha\|_2^2 \leq 72
\lambda\|x\|_1, \quad \|\alpha\|_\infty \leq 4\lambda.$$
\end{lemma}
  
\begin{proof}
请注意,对于$k \geq 1$,有$d\alpha_k =
\pi_{k-1}dy_k\pi_{k-1}$。因此,$L_1$估计直接由~\eqref{Est4}~和引理~\ref{marting-y}~得出。对于$L_\infty$估计,我们回顾一下~\eqref{Ident1},对于每个$n\geq 1$,
\[
\alpha_n= \sum_{k=1}^{n} \pi_{k-1} dy_k  \pi_{k-1} = \pi_n  y_n
\pi_n + \sum_{k=1}^{n} (\pi_{k-1}-\pi_{k}) y_k
(\pi_{k-1}-\pi_{k}).
  \]
关键观察是$\sup_{k\geq 1} \|dy_k\|_\infty \leq 2
\lambda$。我们有以下估计:
\begin{eqnarray*}
\left\|\alpha_n \right\|_\infty & \leq & \left\|\pi_n  y_n \pi_n
\right\|_\infty \\ & + & \Big\| \sum_{k=1}^{n} (\pi_{k-1}-\pi_{k})
dy_{k} (\pi_{k-1}-\pi_{k}) \Big\|_\infty
\\ & + & \Big\|\sum_{k=1}^{n}
(\pi_{k-1}-\pi_{k}) y_{k-1} (\pi_{k-1}-\pi_{k}) \Big\|_\infty.
\end{eqnarray*}
由$(\pi_n)_{n \ge 0}$的定义可推出,
$$\left\|\alpha_n \right\|_\infty \leq \lambda + \sup_{k \leq n}
\left\|dy_{k}\right\|_\infty + \sup_{k\leq n}
\left\|(\pi_{k-1}-\pi_{k}) y_{k-1}
(\pi_{k-1}-\pi_{k})\right\|_\infty \leq 4\lambda.$$ 通过应用H\"older不等式,$L_2$估计可以从$L_1$和$L_\infty$估计得出。
\end{proof}

\begin{lemma}\label{beta-norm}
鞅 $\beta$ 满足 $$\sum_{k=1}^\infty
\|d\beta_k\|_1 \leq 7\|x\|_1.$$
\end{lemma}

\begin{proof}
根据 $(d\beta_k)_{k \ge 1}$ 的定义,我们有
\[
\sum_{k=1}^{\infty} \left\|d\beta_k\right\|_1 \le
\left\|d\beta_1\right\|_1 + \sum_{k=2}^{\infty} \big\| q_{k-1}
dx_k q_{k-1} - q_k dx_k q_k \big\|_1 + \sum_{k=2}^{\infty} \big\|
\E_{k-1} \big( q_k dx_k q_k \big) \big\|_1.
\]
因为对每个 $k\geq 2$, $$\E_{k-1} \big( q_k dx_k q_k
\big) = \E_{k-1} \big( q_k dx_k q_k - q_{k-1} dx_k q_{k-1}
\big),$$  我们从~\eqref{Est3} 推断
\[
\sum_{k=1}^{\infty} \left\|d\beta_k\right\|_1 \le
\left\|x_1\right\|_1 + 2 \sum_{k=2}^{\infty} \big\| q_{k-1} dx_k
q_{k-1} - q_k dx_k q_k \big\|_1 \le 7 \|x\|_1.
\]
因此我们有如上所述的估计。
\end{proof}

\begin{lemma}\label{gamma-support}
对每个 $k\geq 1$,
\begin{itemize}
\item[(a)]$\mathrm{supp} |d\gamma_k| \leq {\bf 1}- \pi_{k-1}
\wedge q_{k-1}$;
\item[(b)] $\mathrm{supp} |d\upsilon_k^*| \leq {\bf 1}-
\pi_{k-1} \wedge q_{k-1}$.
\end{itemize}
因此,鞅 $\gamma$ 和 $\upsilon$ 满足
$$\max \Big\{ \lambda \tau \Big( \bigvee_{k \ge 1} \mathrm{supp}
\, |d\gamma_k| \Big), \, \lambda \tau \Big( \bigvee_{k \ge 1}
\mathrm{supp} \, |d\upsilon_k^*| \Big) \Big\} \le 10 \, \|x\|_1.$$
\end{lemma}
  
\dem
通过~\eqref{decomposition},我们立即得到$d\gamma_k=d\gamma_k({\bf 1}-\pi_{k-1}\wedge q_{k-1})$,并且使用极分解,我们得到$|d\gamma_k|=|d\gamma_k|({\bf
1}-\pi_{k-1}\wedge q_{k-1})$,这表明$\mathrm{supp} \,
|d\gamma_k| \leq {\bf 1}- \pi_{k-1} \wedge q_{k-1}$。因此,
\[
\bigvee_{k\geq 1} \mathrm{supp} \, |d\gamma_k| \leq {\bf 1}- \pi
\wedge q.\] 因此,我们推导出
$$\T \Big( \bigvee_{k\geq 1}
\mathrm{supp} \, |d\gamma_k| \Big) \leq \T({\bf 1}- \pi) + \T({\bf
1}-q) \leq \frac{1}{\lambda} \big( \|y\|_1 + \|x\|_1 \big) \leq
\frac{10}{\lambda} \, \|x\|_1.$$ 同样的论证适用于鞅差序列$(d\upsilon^*_k)_{k \ge 1}$。
\fin
现在很明显,通过结合引理~\ref{alpha-norm}、引理~\ref{beta-norm}和引理~\ref{gamma-support},定理~\ref{Main}的(ii)、(iii)和(iv)项的所有估计都得到了验证。此外,关于$\gamma$和$\upsilon$是$L_1$鞅的事实直接从~\eqref{decomposition}~得到。
\fin

%%%%%%%%%%%%%%%%%%%%%%%%%%%%%%%%%%%%%%%%%%%%%%%%%%%%%%%%%%%%%%%%%

\begin{remark}
\emph{需要注意的是,与可交换情况形成鲜明对比的是,
在定理~\ref{Main}中所陈述的分解中,考虑第四个鞅$\upsilon$是必要的。
事实上,假设在定理~\ref{Main}中的分解可以仅使用三个鞅完成。
也就是说,对于每个有界$L_1$鞅$x$和$\lambda>0$,
存在一个分解$x=\alpha +\beta +\gamma$,满足定理~\ref{Main}的(ii)、(iii)和(iv)项。
那么,可以对经典情况中在\cite{G}中使用的论证进行直接调整,
从而证明存在一个绝对常数$\mathrm{c}$使得 $$\max \Big\{
\|\mathcal{S}_R(x)\|_{1,\infty}, \,
\|\mathcal{S}_C(x)\|_{1,\infty} \Big\} \leq \mathrm{c} \,
\|x\|_1.$$ 特别地,实插值方法的标准应用表明存在一个只依赖于$p$的
常数$\mathrm{c}_p$,使得如果$x$是$L_p$有界的鞅,$1 < p < 2$,
则我们有 $$\max \Big\{ \|\mathcal{S}_R(x)\|_{p},
\, \|\mathcal{S}_C(x)\|_{p} \Big\} \leq \mathrm{c}_p \, \|x\|_p.$$ 这与\cite{PX1}中证明的
非交换Burkholder-Gundy不等式的非交换类比直接冲突。
因此,一般情况下,在定理~\ref{Main}中不能将分解为三个鞅。
这一观察结果证实了我们的分解与非交换鞅的Hardy空间的行/列结构之间的关系。
此外,对上述简略论证的详细检查表明,实际上,鞅$\gamma$和$\upsilon$可以被看作是
\cite{G}中其可交换对应物的“ 列部分" 和“ 行部分"。}
\end{remark}

\begin{remark}
\emph{在上述构造~\eqref{decomposition}~中,我们有
$\alpha_1=0$,$\beta_1= x_1$,$\gamma_1=0$和$\upsilon_1=0$。
任何其他选择这些第一项的方式都不会对定理~\ref{Main}中所述的性质产生任何影响。
我们的选择在一定程度上受到下面的第二个应用的启发,
其中我们需要$\gamma_1 = \upsilon_1 = 0$。}
\end{remark}
%%%%%%%%%%%%%%%%%%%%%%%%%%%%%%%%%%%%%%%%%%%%%%%%%%%%%%%%%%%%%%%

在下面的表述中,我们观察到,如果希望在定理~\ref{Main}的分解中使用三个鞅,
那么我们必须考虑一个较弱的支撑投影的概念。

\begin{definition}
对于一个不一定自伴的算子$x \in \M$,我们定义$x$的\emph{双边零化投影}为满足$qxq = 0$的最大投影$q$。在这种情况下,我们设定$\mbox{supp}^* x := {\bf 1} - q$。
\end{definition}

显然,如果$\mathcal{M}$是可交换的,那么$\mbox{supp} \, x = \mbox{supp}^* x$。
一般情况下,$\mbox{supp}^*$比通常的支撑更小,即对于任意的自伴算子$x \in \M$,
我们有$\mbox{supp}^* x \le \mbox{supp} \, x$,
对于非自伴算子$x \in \M$,$\mbox{supp}^* x$是$x$的右支撑和左支撑的子投影。
利用这种支撑投影的概念,我们可以陈述如下结论:

\begin{corollary}
如果$x=(x_n)_{n \ge 1}$是一个$L_1$有界的非交换鞅,$\lambda$是一个正实数,
那么存在三个鞅$a$、$b$和$c$,满足以下性质,其中$\mathrm{c}$是一个绝对常数:
\begin{itemize}
\item[(i)] $x=a+b+c;$
\item[(ii)] 鞅$a$满足 $$\|a\|_1 \leq \mathrm{c}
\|x\|_1, \quad \|a\|_2^2 \leq \mathrm{c} \lambda \|x\|_1, \quad
\|a\|_\infty \leq \mathrm{c} \lambda;$$
\item[(iii)] 鞅$b$满足 $$\sum_{k=1}^{\infty}
\|db_k\|_1 \le \mathrm{c} \|x\|_1;$$
\item[(iv)] 鞅$c$满足
$$\lambda \tau \Big( \bigvee_{k \ge 1} \mathrm{supp}^* dc_k \Big)
\le \mathrm{c} \|x\|_1.$$
\end{itemize}
\end{corollary}

\begin{proof}
根据~\eqref{decomposition},我们只需设定$a:=\alpha$,$b:=\beta$和
$c:=\gamma +\upsilon$。然后,(ii)和(iii)直接由定理~\ref{Main}得出。
对于(iv),我们注意到从~\eqref{decomposition}中,对于每个$k\geq 1$,
$dc_k=dx_k -\pi_{k-1}q_{k-1}dx_kq_{k-1}\pi_{k-1}$。因此,
我们推导出对于$k\geq 1$,$(\pi_{k-1} \wedge q_{k-1})dc_k(\pi_{k-1} \wedge q_{k-1})=0$,
因此,$\mathrm{supp}^* dc_k \leq {\bf 1}-(\pi_{k-1} \wedge q_{k-1})$。
特别地,我们得到 $$\T \Big( \bigvee_{k \ge 1} \mathrm{supp}^* dc_k \Big) \leq \T({\bf 1}-\pi) +\T({\bf 1}-q).$$ 
在这一点上,(iv)类似于引理~\ref{gamma-support}中的证明。证明完成。
\end{proof}
值得注意的是,Gundy在\cite{G}中的原始证明使用了两个停时。
这基本上解释了为什么我们在~\eqref{decomposition}的构造中需要两个步骤。
在\cite{B}中,Burkholder提供了Gundy分解的一个较弱版本,其中只需要第一个鞅的$L_2$估计。
他的方法只使用了一个停时。在下一个结果中,我们提供了Burkholder方法的非交换模拟。
这提供了一个更简单的分解,对于某些应用更有用。

\begin{corollary} \label{Gundy}
设 $x=(x_n)_{n \ge 1}$ 是一个 $L_1$-有界正鞅并且
$\lambda$ 是一个正实数。我们考虑 $x$ 分解为四个鞅的和 $x= \alpha' +
\beta' + \gamma' + \upsilon'$ 其中各自的鞅差如下述
\begin{equation*}\label{Decomposition2}
\begin{cases}
 d\alpha'_k &:= q_k^{(\lambda)} dx_k q_k^{(\lambda)} -
\E_{k-1} \big( q_k^{(\lambda)} dx_k q_k^{(\lambda)} \big), \\
d\beta'_k &:= q_{k-1}^{(\lambda)} dx_k q_{k-1}^{(\lambda)} -
q_k^{(\lambda)} dx_k q_k^{(\lambda)} + \E_{k-1} \big(
q_k^{(\lambda)} dx_k q_k^{(\lambda)} \big), \\ d\gamma'_k &:= dx_k
- dx_k q_{k-1}^{(\lambda)},\\ d\upsilon'_k &:= dx_k
q_{k-1}^{(\lambda)} - q_{k-1}^{(\lambda)} dx_k
q_{k-1}^{(\lambda)}.
\end{cases} \tag{$\mathbf{G}'_{\lambda}$}
\end{equation*}
那么,有如下性质:
\begin{itemize}
\item[(i)] 鞅 $\alpha'$ 满足 $$\|\alpha'\|_1 \le
\mathrm{c} \|x\|_1, \quad \|\alpha'\|_2^2 \leq \mathrm{c}
\lambda\|x\|_1;$$
\item[(ii)] 鞅 $\beta', \gamma'$ 和 $\upsilon'$
与定理~\ref{Main} 中的 $\beta, \gamma$ 和 $\upsilon$ 有同样估计。
\end{itemize}
\end{corollary}

\dem 关于 $\beta', \gamma'$ 和 $\upsilon'$ 的估计可以逐字句在定理~\ref{Main}的证明中验证。 对于 $\alpha'$ 的
$L_1$-估计, 我们利用引理~\ref{marting-y}. 为了估计  $\alpha'$ 的 $L_2$-范数,我们注意由正交性
对于 $n\geq 1$,
\begin{equation*}
\left\|\alpha'_n\right\|_2^2
=\sum_{k=1}^n\left\|d\alpha'_k\right\|_2^2 \leq 4\sum_{k=1}^n
\left\|q_kdx_kq_k\right\|_2^2.
\end{equation*}
另一方面,因为对于每个 $k\geq 1$,
$q_kdx_kq_k=q_k(q_kx_kq_k -q_{k-1}x_{k-1}q_{k-1})q_k$,我们得出
\begin{equation*}
\left\|\alpha'_n\right\|_2^2 \leq 4 \sum_{k=1}^n \left\|q_k x_kq_k
-q_{k-1}x_{k-1}q_{k-1}\right\|_2^2.
\end{equation*}
最后,由 \cite[Lemma~3.4]{R},这蕴含了
$\left\|\alpha'_n \right\|_2^2 \leq 24\lambda\left\|x\right\|_1.$
\fin

\begin{remark}
推论~\ref{Gundy}显然地推广到非正的鞅。
\end{remark}

% \newpage

%%%%%%%%%%%%%%%%%%%%%%%%%%%%%%%%%%%%%%%%%%%%%%%%%%%%%%%%%%%%%%%%%%%%

% \chapter{一级标题}

% 这是中南大学学位论文\LaTeX{}模板,下面的文字主要作用为对重构后的模板样式设置进行测试。
% 测试样例基本覆盖模板设定,包括多级标题的基本样式,段落与缩进距离。

% \section{二级标题}

% \subsection{三级标题}

% \subsubsection{四级标题}

% 一级标题根据学校提供的Word模板要求,三号黑体居中,上下各空一行,章节号空一个汉字,
% 并且每一章节单独起一页,章节号格式应使用阿拉伯数字而非中文汉字。

% 二级标题为小四号黑体,缩进两个汉字。章节号后空一个汉字。

% 三级标题小四号楷体GB2312,字体包含在项目中,同样缩进两个汉字,章节号后空一个汉字。

% 四级标题参照本科学术论文设计样式,分项采取(1)、(2)、(3)的序号。

% 所有标题样式由\cls{undergraduate.cls}模板文件 \cs{ctexset} 进行设置。

% \section{字体}

% 正文字体默认使用小四号宋体,英文为小四号 Times New Romen,各段行首缩进两个汉字

% 中南大学\cite{csu__2020}坐落在中国历史文化名城──湖南省长沙市,占地面积317万平方米,建筑面积217万平方米,跨湘江两岸,依巍巍岳麓,临滔滔湘水,环境幽雅,景色宜人,是求知治学的理想园地。

% 中南大学由原湖南医科大学、长沙铁道学院与中南工业大学于2000年4月合并组建而成。原中南工业大学的前身为创建于1952年的中南矿冶学院,原长沙铁道学院的前身为创建于1953年的中南土木建筑学院,两校的主体学科最早溯源于1903年创办的湖南高等实业学堂的矿科和路科。原湖南医科大学的前身为1914年创建的湘雅医学专门学校,是我国创办最早的西医高等学校之一。中南大学秉承百年办学积淀,顺应中国高等教育体制改革大势,弘扬以“知行合一、经世致用”为核心的大学精神,力行“向善、求真、唯美、有容”的校风,坚持自身办学特色,服务国家和社会重大需求,团结奋进,改革创新,追求卓越,综合实力和整体水平大幅提升。

% 英文字体展示如下:

% TeX (/tɛx, tɛk/, see below), stylized within the system as TEX, is a typesetting system (or a "formatting system") which was designed and mostly written by Donald Knuth\cite{knuth1984texbook} and released in 1978. TeX is a popular means of typesetting complex mathematical formulae; it has been noted as one of the most sophisticated digital typographical systems.


% \subsection{调节字号}

% 可以使用 \cs{zihao}命令来调节字号。

% \begin{tabular}{ll}
%   \verb|\zihao{3} | & \zihao{3}  三号字 English \\
%   \verb|\zihao{-3}| & \zihao{-3} 小三号 English \\
%   \verb|\zihao{4} | & \zihao{4}  四号字 English \\
%   \verb|\zihao{-4}| & \zihao{-4} 小四号 English \\
%   \verb|\zihao{5} | & \zihao{5}  五号字 English \\
%   \verb|\zihao{-5}| & \zihao{-5} 小五号 English \\
% \end{tabular}

% \subsection{调节字体}

% 需要说明的是由于学校写作指导要求的字体部分不可在Linux上使用,即便你的写作过程是在Linux或者macOS上完成的,
% 我们仍\textbf{强烈建议}您在Windows操作系统上编译最终版论文。

% 中文可选字体以及选用指令如下:

% \begin{tabular}{l l}
%   \verb|\songti| & {\songti 宋体} \\
%   \verb|\heiti| & {\heiti 黑体}  \\
%   \verb|\kaiti| & {\kaiti 楷体}
% \end{tabular}

% 我们在模板中通过调整\verb|\newCJKfontfamily|的AutoFakeBold参数来简单实现字体加粗,在正文中你可以使用习惯的\verb|\textbf|指令来加粗对应中文。如果你需要调整字体,也可以组合使用字体选择并加粗,比如\verb|\kaiti\bfseries|

% \textbf{宋体加粗测试},宋体不加粗测试。

% {\kaiti\bfseries 楷体加粗测试。}{\kaiti 楷体不加粗测试。}

% 目前模板并没有按照一些其他模板写法中常见的,重定向加粗和倾斜效果到upright,TODO:后续可能考虑重定义\verb|emph|和\verb|strong|样式。




% \section{模板主要结构}

% 本项目模板的主要结构, 如下表所示:
% % TODO 进一步完善

% \begin{table}[ht]
%   \centering
%   \begin{tabular}{r|l|l}
%     \hline\hline
%     \multicolumn{2}{l|}{csuthesis\_main.tex } & 主文档,可以理解为文章入口。                      \\ \hline
%                                               & info.tex                     & 作者、文章基本信息 \\ \cline{2-3}
%                                               & abstactzh/en.tex             & 中/英文摘要内容    \\ \cline{2-3}
%     \raisebox{1em}{content 目录 }             & subchapters 目录             & 章节内容           \\ \hline
%     \multicolumn{2}{l|}{images 目录}          & 用于存放图片文件                                  \\ \hline
%     \multicolumn{2}{l|}{csuthesis.cls }       & 模板入口                                          \\ \hline\hline
%   \end{tabular}
% \end{table}

% 我们不建议模板使用者更改原有模板的结构,
% 但如果您确实需要,请务必先充分阅读本模板的使用说明并了解相应的\LaTeX{}模板设计知识。
