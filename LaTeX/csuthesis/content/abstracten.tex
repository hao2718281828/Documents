%!TEX root = ../csuthesis_main.tex
\keywordsen{Operator Algebra\ \ Martingale Theory\ \ Gundy's Decomposition}
\begin{abstracten}

% Martingale theory is the probability theory analogue of Fourier series theory in harmonic analysis. In harmonic analysis, we associate an integrable function to a Fourier coefficient sequence; in martingale theory, we associate an integrable to a martingale difference sequence. In harmonic analysis, Fourier basis is an orthogonal sequence; in martingale theory, a martingale difference sequence is an orthogonal sequence. In harmonic analysis, we transform a Fourier coefficient sequence by "multiplying a (fixed) bounded sequence", and reproduce a function, which is just the multiplier theory; in martingale theory, we product a martingale difference sequence by a predictable sequence and reproduce a martingale, which is just the martingale transform theory.

In harmonic analysis, the Calderón-Zygmund Decomposition (CZ-Decomposition) is a classical method used to prove the weak boundedness of many integral operators. Similarly, in martingale theory, the weak boundedness of operators is of great importance. In this context, Gundy's Decomposition serves as a suitable substitute for CZ-decomposition in martingale theory. Unlike the CZ-decomposition, which decomposes an integrable function into two functions, the classical Gundy's decomposition parts a martingale into three martingales. Additionally, the non-commutative Gundy's decomposition decomposes a martingale into four martingales.
% , the abbreviated form is used in the following context
% On the way of learning Operator Algebra theory, as von Neumann algebra theory develops rapidly; especially, after the work of Pisier and Xu \cite{PX1}, non-commutative martingale theory flourished, and plenty of results in classical martingale theory have found their non-commutative analogue.

This essay will discuss Gundy's decomposition, both the classical version and the non-commutative version. We will use stopping times and Cuculescu projection sequence to prove the classical and non-commutative versions, respectively. Additionally, we will explore some applications of each version.
\end{abstracten}