%!TEX root = ../csuthesis_main.tex
% 设置中文摘要
\keywordscn{算子代数\quad 鞅论\quad Gundy 分解}
%\categorycn{TP391}
\begin{abstractzh}

在调和分析中,Calder\'on-Zygmund 分解(后文中简记为 CZ 分解)是用于证明奇异积分算子弱有界性的典型方法,其本质思想即将函数拆分为两个函数,分而治之。在鞅论中,算子的弱有界性也非常重要。在鞅论框架下,Gundy 分解就是 CZ 分解的鞅论合适替代。区别于调和分析中 CZ 分解将可积函数拆分为两个函数,经典 Gundy 分解将一个鞅分解成三个鞅,而非交换 Gundy 分解将一个鞅分解成四个鞅。
% 在算子代数理论的学习中,随着 von Neumann 代数(后文中简记作 VNA)理论的迅速发展;特别地,在 Pisier 和许全华老师的工作 \cite*{PX1} 后,非交换鞅论蓬勃发展,许多经典鞅论的结果都找到了相应的非交换形式。
\par 本文将讨论 Gundy 分解的经典版本和非交换版本。我们将分别利用停时、Cuculescu 投影序列来证明经典版本和非交换版本。此外, Gundy 分解定理。此外,我们还将探讨每个版本的一些应用。
\end{abstractzh}